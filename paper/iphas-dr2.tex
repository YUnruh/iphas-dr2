\documentclass[useAMS,usenatbib]{mn2e}
\usepackage{amssymb,amsmath}
\usepackage[pdftex]{graphicx}
\usepackage{longtable} 
%\usepackage{lscape}
\usepackage{float}
\usepackage{booktabs}
\usepackage{aas_macros}

\def\ha{\mbox{H$\rm \alpha$}}
\def\arcsec{$''$}
\def\arcmin{$'$}
\def\deg{$^{\circ}$}
\def\micron{\mbox{$\mu$m}}

\title[IPHAS Data Release 2]{The Second Data Release 
of the INT Photometric H$\alpha$ Survey 
of the Northern Galactic Plane (IPHAS DR2)}
		
\author[G. Barentsen
et. al]{G. Barentsen$^{1}$\thanks{E-mail:geert@barentsen.be},
H. J. Farnhill$^{1}$,
J. E. Drew$^{1}$,
M. J. Irwin$^{4}$, \newauthor
E. A. Gonz$\acute{\rm{a}}$lez-Solares$^{4}$,
R. Greimel$^{2}$,
B. Mizalski$^{3}$,
C. Ruhland$^{1}$, \newauthor plus many friends.
\newauthor\\
$^{1}$Centre for Astrophysics Research, STRI, University of Hertfordshire, College Lane Campus, Hatfield AL10 9AB\\
$^{2}$Institute for Geophysics, Astrophysics and Meteorology, Institute of Physics, Karl-Franzens-Universit$\ddot{a}$t Graz, Universit$\ddot{a}$tsplatz 5, 8010 Graz, Austria\\
$^{3}$Building, Institute, Street Address, City, Code, Country\\
$^{4}$Institute of Astronomy, Madingley Road, Cambridge CB3 0HA}

\begin{document}
\date{Current draft typeset \today}
\pagerange{\pageref{firstpage}--\pageref{lastpage}} \pubyear{2013}

\maketitle

\label{firstpage}

\begin{abstract} % BACKGROUND, OBJECTIVE, METHODS, RESULTS, CONCLUSIONS
The INT/WFC Photometric H$\alpha$ Survey 
of the Northern Galactic Plane (IPHAS)
is a 1800 deg$^2$ imaging survey
covering the entire northern Milky Way at $|b| < 5^\circ$
in the $r$, $i$ and \ha\ filters 
using the Wide Field Camera (WFC) 
on the 2.5-meter Isaac Newton Telescope (INT).
This data release presents the first 
uniformly-calibrated source catalogue
to have been extracted from the survey,
providing single-epoch photometry
for 219 million unique sources
across 92\% of the survey.
The observations were carried out between 2003 and 2012
at a median seeing of 1.1 arcsec
to a depth of $r=21.2\pm 0.5$, $i=20.0\pm 0.3$ and \ha$=20.3\pm 0.3$
($5\sigma$ limits, Vega system).
We explain the data reduction 
and quality control procedures,
describe and test the new uniform photometric calibration,
and detail the construction of the source catalogue
and its quality warning flags.
We find that the new calibration is accurate to
$\sigma=0.03$ mag.
Finally, we demonstrate the ability of the 
catalogue's unique
$r$-$i$/$r$-\ha\ colour-colour diagram to
(i) characterise stellar populations and extinction regimes
towards different Galactic sightlines
and (ii) select reliable candidate \ha-emission line objects.
The catalogue which accompanies this paper
provides the much-needed visible-light counterpart
to several infrared surveys of the Galactic Plane,
including many poorly-studied star-forming regions and ``bubbles'',
and provides images for some of the most
crowded regions to be faced by Gaia.
\end{abstract}

\begin{keywords}
catalogues, surveys, stars: emission line, Be, Galaxy: stellar content
\end{keywords}

\section{Introduction}

Since the first data release in 2008, 
the INT/WFC Photometric H$\alpha$ Survey 
of the Northern Galactic Plane (IPHAS)
has provided new insights into the contents and structure 
of our own backyard, the Milky Way. 
The original motivation for undertaking 
this large-scale programme of observation
-- spanning almost a decade 
and using more than 300 nights 
at the Isaac Newton Telescope (INT) in La Palma -- 
was to provide the digital update 
to the photographic northern H$\alpha$ surveys 
of the mid-20th century. 
By increasing the sensitivity 
with respect to these previous surveys 
by a factor $\sim$1000 (7 magnitudes), 
it was envisaged that IPHAS would allow 
the limited bright samples of Galactic emission line objects 
available at the outset \citep[e.g.][]{Kohoutek1999}, 
to be extended into larger, deeper, more statistically-robust samples 
that in turn could better inform our understanding 
of the early and late stages of stellar evolution. 
This aim has begun to be realised through a
range of published studies including: 
a preliminary catalogue of candidate emission line objects \citep{Witham2008};
discoveries of new northern symbiotic stars \citep{Corradi2008, Corradi2010}; 
new cataclysmic variables \citep{Witham2007}; 
new groups of young stellar objects \citep{Vink2008,Barentsen2011a};
along with discoveries of new and remarkable planetary nebulae 
\citep{Mampaso2006, Corradi2011, Viironen2011}.

Over the years it has become apparent that the legacy of IPHAS 
will extend beyond the traditional \ha\ applications 
of identifying emission line stars and nebulae. 
Through the provision of $r$, $i$ broadband photometry 
alongside H$\alpha$ data,
IPHAS has created the opportunity 
to study Galactic Plane populations 
in a new way.
For example, the survey’s unique $r-H\alpha$ colour, 
when combined with $r-i$,
has been shown to provide simultaneous constraints 
on intrinsic stellar colour and interstellar extinction \citep{Drew2008}. 
This has opened the door 
to a wide range of Galactic science applications, 
including the mapping of extinction across the Plane in three dimensions
and the probabilistic inference of stellar properties
\citep{Sale2009, Sale2010, Giammanco2011, Sale2012, Barentsen2013}. 
In effect, the availability of narrowband H$\alpha$ 
alongside $r, i$ magnitudes
provides coarse spectral information for huge samples of stars 
which are otherwise too faint or numerous 
to be targeted by spectroscopic surveys.
For such science applications to succeed however, 
it is vital that the imaging data is transformed 
into a homogeneously calibrated photometric catalogue, 
in which quality problems 
and duplicate detections of the same source 
are flagged. 

The first release of IPHAS data, 
covering roughly half the survey footprint,
was made in late 2007 \citep{Gonzalez-Solares2008}. 
At the time the data were insufficiently complete 
to support a homogeneously calibrated source catalogue.
The goal of this paper is to present the next release 
that takes the coverage up to over 90 percent of the survey area 
and includes a uniform calibration.
In this work we aim to
(i) explain the data reduction 
and quality control procedures that were applied,
(ii) describe and test the new global photometric calibration, and 
(iii) detail the construction and demonstrate the use 
of the source catalogue that has been extracted 
from the newly re-calibrated data.

In \S\ref{sec:observations} we start by recapitulating the key points
of the survey design and strategy.
In \S\ref{sec:reduction} we describe the data reduction
and quality control procedures.
In \S\ref{sec:calibration} we explain the uniform re-calibration
and test our results against the Sloan Digital Sky Survey (SDSS).
In \S\ref{sec:catalogue} we explain how the source catalogue was compiled.
In \S\ref{sec:discussion} we discuss the properties of the catalogue,
and finally in \S\ref{sec:demonstration} we demonstrate
the health of the release by demonstrating its scientific exploitation.
In \S\ref{sec:conclusions} we conclude and outline
our future ambitions.


\section{Observations}
\label{sec:observations}

The detailed properties of the IPHAS observing programme 
have been presented before 
by \citet{Drew2005} and \citet{Gonzalez-Solares2008}. 
To set the stage for this release, we briefly remind of some key points.
IPHAS is a 1800~sq. deg. imaging survey of the northern Galactic Plane, 
providing images and photometry in Sloan $r, i$ 
along with narrowband H$\alpha$. 
It is carried out using the Wide Field Camera (WFC) 
on the 2.5-meter Isaac Newton Telescope (INT) in La Palma. 
It is the first digital survey to offer comprehensive CCD photometry
of point sources in the Galactic Plane at visible wavelengths, 
and does so down to a limiting magnitude of $\sim$20th.
The IPHAS footprint on the northern sky spans a box 
of roughly 180 by 10 degrees, 
taking in the entire northern Galactic Plane 
at latitudes $-5^{\circ} < b < +5^{\circ}$ 
and longitudes $30^{\circ} < l < 215^{\circ}$.

The Wide Field Camera is a mosaic of 4 CCDs 
that captures a sky area of close to 0.29 square degrees.
To cover the entire northern Plane with some overlap,
the survey area was divided into 7635 telescope pointings.
Each of these pointings is accompanied by an offset position
at a displacement of $+$5 arcmin in declination 
and $+$5 arcmin in right ascension,
to deal with inter-CCD gaps, detector imperfections,
and to enable quality checks. 
The basic unit of observation hence
amounts to $2 \times 3$ exposures, 
in which each of the 3 survey filters is exposed at 2 offset sky positions, 
within an elapsed time of 10 minutes.
We shall refer to the unit of 3 exposures at the same position 
as a \emph{field},
and the combination of two fields at a small offset as a \emph{field pair}.
The survey hence contains 15270 fields grouped into 7635 field pairs.
To achieve the desired survey depth of 20th magnitude or fainter, 
the filter exposure times were set at 120 sec (narrowband \ha), 
30 sec ($r$) and 10 sec ($i$)
in the majority of the survey observations.\footnote{In 2003 
the $r$-band exposure time was 10~sec instead of 30~sec,
and since Oct 2010 the $i$-band exposure time 
has been increased from 10~sec to 20~sec.}

Data-taking began in the second half of 2003, 
and every field had been observed at least once by the end of 2008.
At that time only 76 percent of the field pairs 
satisfied our minimum quality criteria however,  
often due to the effects of clouds, poor seeing, or technical faults
(the quality criteria will be detailed in the next section). 
Since then, a programme of repeat observations has been in place 
to improve data quality. 
As a result, 92\% of the survey 
now benefits from quality-approved data.
The most recent observations which are part of this release
were obtained in November 2012.

\begin{figure*}
        \includegraphics[width=1\linewidth]{./plots/footprint/footprint_small.png}
        \caption{Survey area showing the footprints
        of all the quality-approved IPHAS fields
        which have been included in this data release.
        The area covered by each field has been coloured black
        with a semi-transparent opacity of 20\%,
        such that regions where fields overlap are darker.
        The IPHAS strategy is to observe each field twice
        with a small offset,
        and hence the vast majority of the area 
        is covered twice (dominant gray colour).
        There are small overlaps between all the neighbouring fields
        which can be seen as a honeycomb-style pattern
        of dark gray lines across the survey area.
        Regions with incomplete data are apparent as white gaps (no data) 
        or as the lightest shade of gray
        (denoting that only the offset position is missing).}
        \label{fig:footprint}
\end{figure*}

Figure~\ref{fig:footprint} shows the footprint 
of the quality-approved observations included in this work. 
The fields which remain missing 
-- covering 7 percent of the survey area --
are predominantly located towards the Galactic anti-center 
at $l > 120^o$.
Fields at these longitudes can only be accessed from La Palma 
in the months of November-December,
which is when the weather and seeing conditions are often poor
at the INT and observing attempts have failed repeatedly.
To enable the survey to be brought to completion, 
a decision was made recently to limit repeats in this area 
to individual fields requiring replacement,
i.e. fresh observations in all 3 filters may only be obtained 
at one of the two offset positions, 
if the data for the partner offset has already passed quality control.  
The catalogue is structured such that it is clear 
where a contemporaneous observation of both halves of a field pair
is not available.


\section{Data reduction and quality control}
\label{sec:reduction}

\subsection{Initial pipeline processing}

All raw data obtained with the INT were transferred
to the Cambridge Astronomical Survey Unit (CASU) 
for initial processing and archival.
The procedures used by CASU were originally devised
for the INT Wide Field imaging Survey \citep[WFS;][]{McMahon2001,Irwin2005},
which was a 200 deg$^2$ survey programme carried out 
between 1998 and 2003 after the WFC was commissioned.
Because IPHAS uses the same telescope and camera combination,
we have been able to benefit from the existing WFS pipeline.
A detailed description of the processing steps 
is found in \citet{Irwin2001}.
Its application to IPHAS has previously been described
by \citet{Drew2005} and \citet{Gonzalez-Solares2008}
and much of the source code is available 
on line\footnote{http://casu.ast.cam.ac.uk/surveys-projects/software-release}. 
In brief, the pipeline takes care of bias subtraction,
linearity correction, flat-fielding,
gain correction and de-fringing.

The reduced images are then stored in a multi-extension FITS file 
with a primary header describing the characteristics
(position, filter, exposure time, etc.) 
and four compressed image extensions 
corresponding to each of the four CCDs.
Source detection and characterisation is then carried out 
using the \textsc{imcore} tool \citep{Irwin1985,Irwin1997}.
The flux of each source is measured using both
the peak pixel height (i.e. a square 0.33$\times$0.33\arcsec\ aperture)
as well as a series of circular apertures of increasing diameter 
(1.2\arcsec, 2.3\arcsec, 3.3\arcsec, 4.6\arcsec\ and 6.6\arcsec).

The local background levels are estimated 
by computing the sigma-clipped median
flux in a grid of 64$\times$64 pixels (21$\times$21\arcsec)
across the image,
which is then interpolated to obtain an estimate 
of the background level at each pixel.
These sky levels are subtracted from the aperture photometry and
-- when required --
a deblending routine is applied which also attempts to remove
the contamination from any other nearby sources.
Whilst this approach works very well 
across the vast majority of the survey area,
the Galactic Plane unavoidably contains crowded regions 
with large numbers of overlapping sources
or rapidly spatially-varying nebulosity,
in which case aperture photometry must always be interpreted 
with caution.
In \S\ref{sec:catalogue} we will explain that overlapping
sources to which the deblending routine was applied 
are flagged in the catalogue using the \emph{deblend} warning flag.

Finally, an astrometric solution is determined
based on the 2MASS point source catalog \citep{Skrutskie2006},
which itself is calibrated 
in the International Celestial Reference System (ICRS).
A provisional photometric calibration is also provided 
based on the average zeropoint
determined from a set of standard stars observed in the same night.
Sources are classified morphologically
-- stellar, galaxy or noise --
based on the curve-of-growth determined
from measuring the source intensity in a series of growing apertures.
Finally, the resulting source detection tables are also stored 
in multi-extension FITS files.

At the time of preparing DR2,
the CASU pipeline had processed
74\,195 IPHAS exposures 
in which a total of 1.9~billion \emph{candidate sources} were detected 
at the sensitive default detection level of 1.25\,$\sigma$
-- unavoidably including spurious detections, artefacts and
duplicate detections 
(in \S\ref{sec:catalogue} we will explain
how these have been removed or flagged).
The pipelined data set -- comprising 2.5~terabyte of FITS files --
was then transferred to the University of Hertfordshire
for the purpose of transforming the raw
detection tables into a reliable source catalogue which is 
(i) quality-controlled,
(ii) homogeneously calibrated, and 
(iii) contains user-friendly columns and warning flags.
It is these post-processing steps which are explained below.


\subsection{Quality control}
\label{sec:qc}

Observing time for IPHAS was obtained
on a semester-by-semester basis
through the traditional time allocation committees 
of the Isaac Newton Group of telescopes,
which are competitive and invariably over-subscribed.
For this reason, we attempted to utilise 
\emph{all} the nights allocated to IPHAS,
even those which were partially or entirely non-photometric
or otherwise affected by technical problems 
(e.g. electronic noise or telescope tracking problems).
Any unsuitable data that was taken as a result
of this strategy was subsequently flagged and rejected
using a series of seven quality criteria,
which ensure a reliable and homogeneous level of quality
across the data release:

\begin{figure}
    \begin{minipage}[b]{\linewidth}
        \includegraphics[width=\textwidth]{./plots/depth_r.pdf} 
    \end{minipage}
    \begin{minipage}[b]{\linewidth}
        \includegraphics[width=\textwidth]{./plots/depth_i.pdf} 
    \end{minipage}
    \begin{minipage}[b]{\linewidth}
        \includegraphics[width=\textwidth]{./plots/depth_h.pdf} 
    \end{minipage}
    \caption{Distribution of the 5$\sigma$ limiting magnitude
             across all quality-approved survey fields
             for $r$ (top), $i$ (middle) and \ha\ (bottom).
             Fields with a limiting magnitude brighter than
             20th ($r$) or 19th (\ha/$i$) were rejected
             from the data release.
             The $r$-band depth is most sensitive 
             to the presence of the moon above the horizon, 
             which is evidenced by the wide and bi-model shape
             of its distribution.}
    \label{fig:depth}
\end{figure}

(1) \emph{Depth.} 
We discarded any exposures for which the $5\sigma$ limiting magnitude 
was worse than 20th magnitude in the $r$-band
or worse than 19th in $i$ or \ha. 
Such data were typically obtained during poor weather or full moon.
Most observations fared significantly better than these limits.
Figure~\ref{fig:depth} presents the distribution of limiting magnitudes
for all quality-approved fields,
which shows a mean depth of 
$21.2\pm0.5$ ($r$), $20.0\pm0.3$ ($i$) and $20.3\pm0.3$ (\ha).
We found that the depth achieved depended 
most strongly on the presence of the moon,
which was above the horizon during 62 percent 
of our observations and explains the wide and bi-modal shape
of the $r$-band limiting magnitude distribution 
(top panel in Fig.~\ref{fig:depth}).
In contrast, the depth attained in $i$ and \ha\ 
is less sensitive to moonlight
and the distribution of their depths
is hence more narrow
(middle and bottom panel in Fig.~\ref{fig:depth}).

Our minimum limiting magnitude criteria
have led us to exclude 9\% of the pipelined data.
We note that much of the excluded data may nevertheless be useful
for e.g. time-domain studies of bright stars.
The detection tables for any such discarded data are made
available through our website (www.iphas.org)
but are ignored in what follows.

(2) \emph{Ellipticity.} 
The ellipticity of a point source,
defined as $e = 1 - b / a$ 
with $b$ the semi-minor and $a$ the semi-major axis,
is a morphological measure of the elongation of the point spread function.
It is expected to be zero (circular) across the field 
in a perfect imaging system,
but it is slightly non-zero in any real telescope data 
due to optical distortions and tracking errors.
The mean ellipticity across a field in the IPHAS data set 
is $0.09\pm0.04$.
There have been sporadic episodes with higher ellipticities however
due to mechanical glitches in the telescope tracking system.
For this reason, we rejected exposures in which the mean ellipticity
across the detector exceeded $e > 0.3$,
which is when the photometric measurements delivered by the pipeline
were found to become degraded.
Only 0.4\% of the exposures were discarded on this basis.

(3) \emph{Seeing.} 
The survey originally aimed to obtain data 
at seeing better than 1.7 arcsec.
This target is currently attained across 86\% of the footprint,
in particular at early longitudes,
e.g. 92\% of the fields at $l<120^\circ$ are better than 1.7 arcsec.
Figure~\ref{fig:seeing} presents the distribution
of the mean seeing for all the quality-approved fields.
We find a median value of 1.1~arcsec in $r$/\ha\
and 1.0~arcsec in $i$.
In the $r$-band, 90\% of the data is better than 1.5~arcsec
and 10\% is better than 0.8~arcsec.
To improve the area covered by this data release,
we have decided to include the small fraction of data
that was obtained under seeing up to 2.5 arcsec,
so that only 1 per cent of the pipelined exposures
had to be excluded.
In \S\ref{sec:catalogue} we will explain
that the catalogue is structured
such that the information on the seeing is included,
and that the photometry listed
is based on the exposures with the best-available seeing.

\begin{figure}
    \begin{minipage}[b]{\linewidth}
        \includegraphics[width=\textwidth]{./plots/seeing_r.pdf} 
    \end{minipage}
    \begin{minipage}[b]{\linewidth}
        \includegraphics[width=\textwidth]{./plots/seeing_i.pdf} 
    \end{minipage}
    \begin{minipage}[b]{\linewidth}
        \includegraphics[width=\textwidth]{./plots/seeing_ha.pdf} 
    \end{minipage}
    \caption{Seeing distribution across all
             quality-approved survey fields
             for $r$ (top), $i$ (middle) and \ha\ (bottom).}
    \label{fig:seeing}
\end{figure}


(4) \emph{Photometric repeatability.} 
The IPHAS field-pair observing strategy 
ensures that every pointing is immediately followed 
by an offset pointing at a displacement of $+$5 arcmin in Dec 
and $+$5 arcmin in RA.
This allows pairs of images to be checked 
for the presence of clouds or electronic noise.
To exploit this information,
the overlap regions of all field pairs were systematically cross-matched
to verify the consistency of the photometric measurements
for stars observed in both pointings.
We rejected field pairs in which more than 2\% of the stars 
showed an inconsistent measurement at the level of 0.2 mag,
or more than 26\% were inconsistent at the level of 0.1 mag.
These limits were determined empirically by inspecting
the images and photometry by eye.
11\% of the data was rejected as part of this step.

(5) \emph{Source density mapping.}
Spatial maps showing the number density of the detected sources
down to 20th magnitude were created to verify the health
of the data and to check for unexpected artefacts.
%(e.g. bright satellite trails)
In particular, we created density maps
which showed the number of \emph{unique} sources
obtained by cross-matching the detection tables of
all three bands with a maximum matching distance of 1 arcsec.
The success of such a cross-matching procedure crucially depends
on the accuracy of the astrometric solution in each band,
and hence we were able to detect and correct
exposures with poor astrometry by inspecting the density map
for spurious density variations.

(6) \emph{Visual examination.}
All images and their associated photometric colour/magnitude diagrams
were inspected by a team of 20 survey team members, 
such that each image in the data release 
was looked at by at least three different pairs of eyes.
Images deemed unsuitable were flagged, investigated and
excluded from the release if necessary. 
6\% of the attempts to observe a field
were placed on a \emph{black-list}
for this reason, most commonly due to the obvious presence
of clouds or extreme levels of scattered moonlight.
Such fields were often rejected as part of 
one or more of the above quality criteria as well.

(7) \emph{Contemporaneous field data.} 
Finally, only exposures which are part of a sequence 
of three consecutive images of the same field (H$\alpha$/$r$/$i$) 
were considered for inclusion in the release. 
This ensures that the three bands for a given field
are observed at nearly the same time --  
essentially always within 5 minutes.
An exception was made for 9 fields where the three exposures 
could not be obtained within the same night
but for which the time gap did not exceed 48 hours.
We note that the exact epoch of the magnitude in each band
is included in the source catalogue
(columns \emph{rMJD}, \emph{iMJD}, \emph{haMJD}).

The above criteria were satisfied 
for 14115 out of the 15270 fields (92\%).
In a few cases more than one successful attempt to observe
a field was available due to the fact that slightly stricter
quality criteria were adopted in the initial years of the survey.
In such cases, only the attempt 
with the best seeing and depth was selected
for inclusion in the release, because the focus 
of this work is to deliver the most reliable
measurement at a single epoch.
Those interested in any of the rejected data 
may nevertheless access the full set of detection tables 
on line.


\section{Photometric calibration}
\label{sec:calibration}

Having obtained a quality-approved set of observations,
we now turn to the problem of placing the data
onto a uniform photometric scale.

\subsection{Provisional nightly calibration}

For the purpose of providing an initial calibration 
of the $r$ and $i$ broadband fluxes,
photometric standard fields were observed every night.
The standards were chosen from a list based on 
the \cite{Landolt1992} and Stetson (http://cadcwww.dao.nrc.ca/standards) 
objects.
Two or three standard fields were observed 
during the evening and morning twilight,
and at intervals of 2-3 hours throughout the night.
The CASU pipeline automatically identified the observed standards 
and used them to determine a sigma-clipped average zeropoint \textsc{magzpt}
for each night and filter,
such that the number counts $DN$ 
in the pipeline-corrected CCD frames
relates to a magnitude $m$ as:
\begin{equation}
\begin{split}
   m  = & \textsc{magzpt} - 2.5 \log_{10}( DN / \textsc{exptime} ) \\
 &  - \textsc{extinct}\cdot(\textsc{airmass}-1) - \textsc{apcor} - \textsc{percorr}
\label{eqn:mag}
\end{split}
\end{equation}
where \textsc{exptime} is the exposure time in seconds,
\textsc{extinct} is the atmospheric extinction coefficient 
(typically 0.09 for $r$ and 0.05 for $i$ in La Palma),
\textsc{airmass} is the normalised optical path length 
through the atmosphere and
\textsc{apcor} is a correction for the flux
lost outside of the aperture used.
Finally, \textsc{percorr} is a correction based on the difference
between the median dark sky for a CCD against the median for all the CCDs 
and as such is an ancillary correction 
to account for sporadic gain variations. 
All these quantities correspond to header keywords in the 
multi-extension FITS files produced by the CASU pipeline.

The zeropoint was determined such that the resulting magnitude system
is in the WFC system that uses the SED of Vega 
as the zero colour reference object. 
Colour equations were used to transform between the IPHAS passbands 
and the Johnson-Cousins system 
of the published standard star photometry.
The entire procedure has been found to deliver zeropoints which 
are accurate at the level of 1-2\% 
in photometric conditions \citep{Gonzalez-Solares2011}.

Unlike the broadbands, 
standard star photometry is not available in the literature 
for the H$\alpha$ passband
and hence there is no formally recognised flux scale 
for the narrowband.
We can specify here however 
that the integrated in-band energy flux for Vega 
in the IPHAS \ha\ filter 
is $1.52 \times 10^{-7}$ ergs\,cm$^{-2}$\,s$^{-1}$ 
at the top of the Earth's atmosphere,
which is the flux obtained by folding 
Vega's SED with the filter throughput curve 
corrected for atmosphere and detector quantum efficiency
\citep[using the method explained by][]{Drew2005}.
This is 3.08 magnitudes less than the flux captured 
by the much broader $r$ band
which includes the \ha\ band.
Hence to assure zero colour relative to the broadbands,
we set the default zeropoint for the narrowband to be:
\begin{equation}
\textsc{magzpt}_{H\alpha} = \textsc{magzpt}_r - 3.08.
\label{eqn:zpha}
\end{equation}

\subsection{Uniform re-calibration}

Despite the best efforts made to obtain a nightly calibration,
large surveys naturally possess field-to-field variations
at the level of 0.1 mag
due to atmospheric fluctuations during the night
and imperfections in the pipeline or the instrument
(e.g. the WFC is known to suffer from sporadic glitches
in the timing of exposures).
Such variations need to be corrected for 
during a global re-calibration procedure.
Notable past examples include the global re-calibration 
of the Two Micron All Sky Survey \citep[2MASS;][]{Nikolaev2000},
the Sloan Digital Sky Survey \citep[SDSS;][]{Padmanabhan2008}
and the Panoramic Survey Telescope 
and Rapid Response System survey \citep[Pan-STARRS;][]{Schlafly2012},
which all achieved photometry 
that is globally consistent to within 1--2\%.

Surveys which observe identical stars at different epochs
can use the repeat measurements to ensure a uniform calibration.
For example, 2MASS attained its global calibration
by observing six standard fields each hour, 
allowing zeropoint variations to be tracked 
over very short timescales \citep{Nikolaev2000}.
Alternatively, the SDSS and PanSTARRS surveys could benefit
from revisiting regions in their footprint to 
carry out a so-called \emph{ubercalibration} procedure,
in which multiple measurements of stars at different epochs
are used to fit the calibration parameters
\citep{Padmanabhan2008,Schlafly2012}.

Unfortunately these schemes cannot be applied
directly to IPHAS
for two reasons. 
Firstly, the survey was carried out 
in competitive observing time
on a non-dedicated telescope, 
rendering the 2MASS approach 
of observing standards at a very high frequency
prohibitively expensive
-- in part because standard fields 
are very scarce within the Galactic Plane.
Secondly, the aim of IPHAS is to obtain magnitudes at a single epoch
and hence stars are not normally observed at more than one epoch,
unless they happen to fall on the narrow overlap region 
between two neighbouring fields.

Although the IPHAS data does contain a significant number of 
inter-field repeat measurements
due to the small overlaps
between neighbouring fields,
we have found the information contained
in these regions to be insufficient
to constrain the calibration parameters.
This is because photometry at the extreme edges of the WFC
-- where neighbouring fields overlap -- 
is itself prone to systematics at the level of 1-2\%.
The cause of these errors is thought to include 
the use of twilight sky flats in the pipeline,
which are known to be imperfect for calibrating stellar photometry 
due to stray light and vignetting \citep[e.g.][]{Manfroid1995}.
Moreover, the illumination correction in the overlap regions
is affected by a radial geometric distortion,
which causes the pixel scale towards the edges 
to increase \citep{Gonzalez-Solares2011}.
Although these systematics are reasonably small within a single field,
they can combine to cause artificial zeropoint gradients 
across the survey
when they are used to constrain a global calibration
without external constraints.

For these reasons, we have decided not to depend
on an ubercalibration-type scheme alone,
and have instead opted to exploit an external reference survey
-- where available --
to bring the majority of our data onto a uniform calibration.
This is explained next.

\subsubsection{Correcting zeropoints using APASS}

We have been able to benefit from the
AAVSO Photometric All-Sky Survey
(APASS; http://www.aavso.org/apass)
to bring the vast majority of the survey 
onto a uniform scale.
Since 2009,
APASS has been using two 20~cm-astrographs
to survey the entire sky down to $\sim$17th magnitude
in five filters which include Sloan $r$ and $i$.
The most recent catalogue available 
at the time of preparing this work was APASS DR7,
which provided data across roughly half of the IPHAS footprint (Fig.~\ref{fig:apass_r}).
The photometric accuracy of APASS is currently estimated 
to be at the level of 3\% (Henden, private communication),
which is significantly better 
than the provisional calibration of IPHAS
for which we estimate the $1\sigma$-error to be $\sim$10\%.
APASS achieves its uniform accuracy 
by measuring each star at least two times in photometric conditions
-- along with ample standard fields --
using the large $3\times3$ square degrees field of view of its detectors.

With the aim of bringing IPHAS to a similar accuracy of $\sim$3\%,
we used the APASS catalogue to identify and adjust all IPHAS fields 
which showed a magnitude offset larger than 3\% against APASS.
For this purpose,
the $r$- and $i$-band detection tables of each IPHAS field
were cross-matched against the APASS DR7 catalogue 
using a maximum matching distance of 1~arcsec.
The magnitude range was limited to
$13<r_{\rm APASS}<16.5$ and $12.5<i_{\rm APASS}<16.0$
in order to avoid sources 
brighter than the IPHAS saturation limit on one hand, 
and to avoid sources near the faint detection limit of APASS 
on the other hand.

The resulting set of $\sim$220\,000 cross-matched stars were then used 
to derive APASS-to-IPHAS magnitude transformations
using a linear least-squares fitting routine, 
which iteratively removed $3\sigma$-outliers to improve the fit.
The solution converged to:
\begin{align} 
r_{\rm IPHAS} = r_{\rm APASS} - 0.121 + 0.032(r-i)_{\rm APASS} \label{eqn:apass_r} \\
i_{\rm IPHAS} = i_{\rm APASS} - 0.364 + 0.006(r-i)_{\rm APASS} \label{eqn:apass_i}
\end{align}
The root mean square (rms) residuals of these transformations 
are 0.041 and 0.051, respectively.
The small colour terms in the equations
indicate that the $r$ and $i$ filters 
are very similar in both surveys.
The transformations include a large fixed offset
which is simply due to the fact that 
APASS magnitudes are given in the AB system
and IPHAS uses magnitudes relative to Vega.
Separate transformations were derived for sightlines 
with varying extinction properties to investigate the robustness
of the transformations with respect to different reddening regimes,
but the variations at these different
sightlines were found to be insignificant.
This is not surprising because heavily reddened objects 
are naturally scarce at $r<16$.

Having transformed APASS magnitudes into the IPHAS system,
we then computed the median magnitude offset 
for each field which contained at least 30 cross-matched stars.
This was the case for 48\% of the fields.
The mean offset was found to be
$0.014\pm0.104$ for $r$ and $0.007\pm0.108$ for $i$
(Table~\ref{tbl:offsets_before}).
A total of 4596 fields showed a median offset
exceeding $\pm$0.03~mag in either $r$ or $i$.

We then applied the most important step in our re-calibration scheme,
which is to adjust the provisional zeropoints of the 4596 aberrant fields
such that their offset is brought to zero.
This allowed the mean IPHAS-to-APASS offset 
to be brought down to $0.000\pm0.011$ in both $r$ and $i$
(Table~\ref{tbl:offsets_after}).
The procedure of fitting magnitude transformations and
correcting the zeropoints was repeated a few times to ensure 
convergence, which was essentially reached after the first iteration.

\begin{figure*}
    \includegraphics[width=\textwidth]{plots/calibration/APASS-IPHAS-DR2_rshift.pdf} 
    \includegraphics[width=\textwidth]{plots/calibration/colourbar_apass_r.pdf} 
    \caption{Median magnitude offset in the $r$ band between IPHAS and APASS,
             plotted on a field-by-field basis
             prior to the re-calibration procedure.
             Each square represents an IPHAS field
             which contains at least 30 stars with a counterpart
             in the APASS DR7 catalogue.
             The colours denote the median
             IPHAS-APASS magnitude offset in each field,
             which was computed after applying the APASS-to-IPHAS
             transformation to the APASS magnitudes (Eqn.~\ref{eqn:apass_r}).}
        \label{fig:apass_r}
    \vspace{1cm}
    \includegraphics[width=\textwidth]{plots/calibration/SDSS-IPHAS_rshift.pdf}
    \includegraphics[width=\textwidth]{plots/calibration/colourbar_sdss_r.pdf} 
    \caption{Median magnitude offset in the $r$ band
             between IPHAS and SDSS after the re-calibration
             procedure was applied.
             Again, each square represents an IPHAS field
             which contains at least 30 stars
             with a counterpart in the SDSS catalogue.
             The colours denote the median IPHAS-SDSS magnitude offset
             in each field,
             which was computed after applying the SDSS-to-IPHAS
             transformation to the SDSS magnitudes (Eqn.~\ref{eqn:sdss_r}).}
    \label{fig:sdss_r}
\end{figure*}

\begin{figure*}
	\vspace{2cm}
    \includegraphics[width=\textwidth]{plots/calibration/APASS-IPHAS-DR2_ishift.pdf} 
    \includegraphics[width=\textwidth]{plots/calibration/colourbar_apass_i.pdf} 
    \caption{Same as Figure~\ref{fig:apass_r} for the $i$-band.}
        \label{fig:apass_i}
    \vspace{1cm}
    \includegraphics[width=\textwidth]{plots/calibration/SDSS-IPHAS_ishift.pdf}
    \includegraphics[width=\textwidth]{plots/calibration/colourbar_sdss_i.pdf} 
    \caption{Same as Figure~\ref{fig:sdss_i} for the $i$-band.}
    \label{fig:sdss_i}
\end{figure*}

\begin{table}
        \begin{center}
                \begin{tabular}{lcc}
                        \toprule
                         {\bf Before re-calibration} & Mean & $\sigma$  \\
                        \midrule
                        $r$ (IPHAS - APASS) & +0.014 & 0.104 \\
                        $i$ (IPHAS - APASS) & +0.007 & 0.108 \\
                        $r$ (IPHAS - SDSS) & +0.016 & 0.088 \\
                        $i$ (IPHAS - SDSS) & +0.010 & 0.089 \\
                        \bottomrule
                \end{tabular}
        \caption{Mean magnitude offsets for objects
                 crossmatched between IPHAS and APASS/SDSS
                 before the uniform re-calibration.
                 Eqns.~\ref{eqn:apass_r}-\ref{eqn:sdss_i} were applied
                 to the APASS/SDSS magnitudes to bring them into the
                 Vega-based IPHAS system prior to computing
                 the offsets.)
                 }
        \label{tbl:offsets_before}

                \begin{tabular}{lcc}
                        \toprule
                         {\bf After re-calibration} & Mean & $\sigma$ \\
                        \midrule
                        $r$ (IPHAS - APASS) & +0.000 & 0.011\\
                        $i$ (IPHAS - APASS) & +0.000 & 0.011 \\
                        $r$ (IPHAS - SDSS)  & -0.001 & 0.029\\
                        $i$ (IPHAS - SDSS) & -0.002 & 0.032 \\
                        \bottomrule
                \end{tabular}
        \caption{Same as Table~\ref{tbl:offsets_before}
                 but computed after the uniform re-calibration
                 was carried out.}
        \label{tbl:offsets_after}
        \end{center}
\end{table}

\subsubsection{Adjusting fields not covered by APASS}

At the time of writing, the APASS catalogue did not provide 
sufficient coverage for 7359 of the fields in our data release.
Fortunately, these fields are pre-dominantly located 
in the early part of the Galactic Plane (Fig.~\ref{fig:apass_r}),
which were typically observed during the summer months
when photometric conditions are more prevalent at the telescope.
These remaining fields have nevertheless 
been brought onto the same uniform scale 
by employing an ubercalibration-style scheme, explained below,
which minimises the magnitude offsets between stars
located in the overlap regions with neighbouring fields.
Although we explained earlier that these overlap regions
are prone to systematics, the use of APASS enabled us to
keep the zeropoints of roughly half of the fields fixed,
which avoids these systematics from combining to introduce
artificial gradients across the survey.
In the next section we will show this to be true by validating
our calibration against SDSS.

A general solution to the problem of minimising the magnitude
differences between overlapping frames has previously been
described by~\citet{Glazebrook1994}.
In brief, there are two fundamental quantities to be
minimised between each pair of overlapping exposures, 
denoted by the indices $i$ and $j$. 
Firstly, the mean magnitude difference between stars in the overlap
region $\Delta_{ij}=\langle m_i-m_j\rangle=-\Delta_{ji}$ is a local
constraint. 
Secondly, to ensure the solution does not stray far 
from the existing calibration, 
the difference in zeropoints 
$\Delta\mathrm{ZP}_{ij}=-\Delta\mathrm{ZP}_{ji}$ 
between each pair of exposures must also be minimised.

Minimisation of these two quantities is a linear least squares problem 
because the magnitude $m$ depends linearly on the ZP (Eqn.~\ref{eqn:mag}).
Hence we can find the ZP shift to be applied to each field 
by minimising the sum:
\begin{equation}
   S = \sum_{i=1}^N \sum_{j=1}^N w_{ij} \theta_{ij} (\Delta_{ij} + a_i - a_j)^2
   \label{eqn:chi2}
\end{equation}
where $i$ denotes the exposure of interest, 
$j$ an overlapping exposure, 
$N$ the number of exposures,
$a_i$ the ZP to solve for 
and $a_j$ the ZP of an overlapping field ($\Delta\mathrm{ZP}_{ij}=a_i-a_j$), 
$w_{ij}$ are weights set to the uncertainty in $\Delta_{ij}$
and $\theta_{ij}$ is an overlap function 
equal to either 1 if exposures $i$ and $j$ overlap or 0 otherwise. 
Solving for $a_i$ is equivalent to solving $\partial
S/\partial a_i=0$ which gives the matrix equation:
\begin{equation}
   \sum_{j=1}^N A_{ij} a_j = b_j
   \label{eqn:matrix}
\end{equation}
where 
\begin{eqnarray}
   A_{ij} &=& \delta_{ij} \sum_{k=1}^N w_{jk}\theta_{jk} - w_{ij} \theta_{ij},\label{eqn:aij}\\
   b_i &=& \sum_{j=1}^N w_{ij} \theta_{ij}\Delta_{ji} = -\sum_{j=1}^N w_{ij} \theta_{ij}\Delta_{ij}.\label{eqn:bi}
\end{eqnarray}
This prescription is essentially identical to \citet{Glazebrook1994}.
%,although we changed the formulation to ensure that $A_{ij}$ is positive definite.

As explained above, we enforce a strong external constraint
on the solution by keeping the zeropoint fixed 
for the 6756 fields which have been compared
and calibrated against APASS earlier.
We will hereafter refer to these fields as \emph{anchors}.
It is asserted that the zeropoints $a_i$ of these anchor fields 
are known and not solved for,
though they do appear in the vector $b_j$ as constraints.
In addition to the APASS-based anchors, 
we selected 3273 additional anchor fields by hand
to provide additional constraints in regions not covered by APASS.
These extra anchors were deemed to have reliable zeropoints 
based on 
(i) the information contained in the observing logs,
(ii) the stability of the standard star zeropoints during the night, and
(iii) photometricity statistics provided by the Carlsberg Meridian Telescope,
which is located at $\sim$500~meter from the INT.

We then solved Eqn.~\ref{eqn:matrix} for the $r$ and $i$ bands
separately using the least-squares routine 
in Python's {\sc scipy.sparse} module for sparse matrix algebra.
This provided us with corrected zeropoints for the remaining fields,
which were shifted on average by $+0.02\pm0.11$ in $r$ 
and $+0.01\pm0.12$ in $i$ compared to their provisional calibration.

We then turned to the uniform calibration of the \ha\ data.
It is not possible to re-calibrate the narrowband 
in the same way as the broadbands,
because the APASS survey does not offer \ha\ photometry.
We can reasonably assume however,
that the corrections required for $r$ and \ha\ are identical
because the \ha\ zeropoints have been derived directly from the
$r$-band zeropoints during the provisional calibration (Eqn.~\ref{eqn:zpha}).
Moreover, the IPHAS data-taking pattern ensured 
that a field's \ha\ and $r$-band exposures
were taken at essentially the same time, 
separated only by the $\sim$30-second read-out time required by the WFC.
We have hence corrected the \ha\ zeropoints 
by re-using the zeropoint adjustements that were derived for the $r$ band
in the earlier steps.
An exception was made for 3101 fields
for which our quality-control routines revealed
zeropoint variations which exceeded 0.03~mag 
between consecutive fields,
which indicates non-photometric conditions.
For good practice, the \ha\ zeropoints of these fields
were adjusted by solving Eqn.~\ref{eqn:matrix}
rather than linking them directly to the $r$-band shift.

\subsection{Testing the calibration against SDSS}

Having re-calibrated all fields to an expected accuracy of 3\%,
we then used an independent survey to validate the results.
We have been able to exploit SDSS Data Release 9 \citep{Ahn2012}
for this purpose.
SDSS DR9 provides several strips at low
Galactic latitudes,
providing data across 18\% of the fields in our data release.
We cross-matched IPHAS fields against 
stars marked as reliable in the SDSS catalogue\footnote{
We used the CasJobs facility located at http://skyserver.sdss3.org/CasJobs
to obtain photometry from the SDSS {\sc photoprimary} table 
with criteria {\sc type = star}, {\sc clean = 1} and {\sc score $>$ 0.7}.}
in the same way as we did for APASS earlier,
with the exception of using fainter magnitude ranges of 
$15<r_{\rm SDSS}<18.0$ and $14.5<i_{\rm SDSS}<17.5$.
This provided us with a set of 1.2 million cross-matched stars.

Colour transformations were obtained using a sigma-clipped linear least squares fit:
\begin{eqnarray}
r_{\rm IPHAS} = r_{\rm SDSS} - 0.093 - 0.044(r-i)_{\rm SDSS} \label{eqn:sdss_r} \\
i_{\rm IPHAS} = i_{\rm SDSS} - 0.318 - 0.095(r-i)_{\rm SDSS}. \label{eqn:sdss_i}
\end{eqnarray}
The rms residuals of these transformations are 0.045 and 0.073, respectively.
The equations are similar to the ones
previously determined for APASS,
although the colour terms are slightly larger;
the throughput curve of the SDSS $i$-band filter 
appears to be somewhat more sensitive at longer wavelengths
compared to both the IPHAS and APASS filters.

These global transformations were deemed adequate
for the purpose of validating our uniform calibration in a statistical sense.
Separate equations were derived towards different sightlines
to investigate the effects of varying reddening regimes.
The transformation coefficients were found 
to show some variation towards lowly reddened areas,
which have relatively few numbers
of (intrinsically) red objects at $r-i > 1$ 
which can skew the colour term.
The vast majority of red objects in the global sample
are those in highly reddened areas however,
which agree well with the global transformations
and dominate the statistical appraisal of our calibration.
%The small number of intrinsically red objects in lowly reddened fields
%have no impact on the statistical appraisal of our calibration.

Having transformed SDSS magnitudes into the IPHAS system,
we then computed the median magnitude offset for each IPHAS field
which contained at least 30 objects with a counterpart
in the SDSS catalogue.
This was the case for 2602 fields.
The median offsets for each of these fields
are shown in Figs.~\ref{fig:sdss_r}-\ref{fig:sdss_i}.
The mean offset and standard deviation found 
was $-0.001\pm0.029$ for $r$
and $-0.002\pm0.032$ for $i$ (Table~\ref{tbl:offsets_after}).
In comparison, offsets computed in the identical way
\emph{before} our re-calibration showed means
of $+0.016\pm0.088$ and $+0.010\pm0.089$ (Table~\ref{tbl:offsets_after}).
We conclude that our re-calibration procedure has
been successful in improving the
uniformity of the calibration by a factor three
and has achieved our aim of bringing the
accuracy down to the level of $\sigma=0.03$~mag.

We warn that the SDSS comparison reveals a number of fields with offsets
exceeding 0.05~mag (523 fields) or even 0.1~mag (18 fields).
Such sporadic outliers are consistent with the tails of a Gaussian
with mean $\sim0$ and $\sigma=0.03$.

In future work, we hope to draw upon
the PanSTARRS survey \citep{Schlafly2012}
to further improve the accuracy of our calibration.
At the time of preparing this work data from PanSTARRS
had not been made public yet.

\section{Source catalogue generation}
\label{sec:catalogue}

Having obtained a quality-checked 
and re-calibrated data set, 
we now turn to the problem
of transforming the observations 
into a user-friendly catalogue.
The aim of this catalogue is to detail
the best-available information for each unique source
in a convenient format,
including flags to warn about quality issues 
such as source blending and saturation.
Compiling the catalogue essentially required four steps:
\begin{enumerate}
\item the single-band detection tables 
produced by the CASU pipeline 
were augmented with new columns
and warning flags;
\item the detection tables were merged into multi-band field catalogues;
\item the overlap regions of the field catalogues 
were cross-matched to flag duplicate measurements 
and identify the best detection 
of each unique source; and
\item these primary detections
were compiled into the final source catalogue.
\end{enumerate}
Each of these four steps are now explained.

\subsection{User-friendly columns and warning flags}

As the first step, the detection tables 
were enhanced by creating new columns.
This is necessary because the tables 
generated by the CASU pipeline 
summarise the detections 
in their original CCD units,
e.g. source positions are given in pixel coordinates 
and photometry in number counts.
To transform these measurements into
user-friendly fields,
we have largely adopted the units and naming conventions
which are in use at the 
WFCAM Science Archive \citep[WSA;][]{Hambly2008}
and the VISTA Science Archive \citep[VSA;][]{Cross2012}.
These archives curate the near-infrared data from both
the UKIDSS Galactic Plane Survey \citep[GPS;][]{Lucas2008}
and the 
VISTA Variables in the Via Lactea survey \cite[VVV;][]{Minniti2010}.
Both these surveys provide high-resolution JHK photometry
in the Galactic Plane.
There is a significant degree of overlap
between the footprints of UKIDSS/GPS and IPHAS,
and hence by adopting a similar catalogue format
we hope to encourage scientific applications
which combine both data sets.

A detailed description of each column in our source catalogue
is given in Appendix~\ref{app:columns}.
In the remainder of this section we highlight the main features.

Firstly, we note that each source is uniquely identified by an
IAU-style designation of the form ``IPHAS2\ JHHMMSS.ss+DDMMSS.s''
(cf. column \emph{name} in Appendix~\ref{app:columns}),
where ``IPHAS2'' refers to the present
data release and the remainder of the string
denotes the J2000 coordinates in sexagesimal format.
For convenience, the coordinates
are also included in decimal degrees
(columns \emph{ra} and \emph{dec})
and in the Galactic coordinate system
(columns \emph{l} and \emph{b}).
We have also included an internal object identifier string 
of the form ``\#run-\#ccd-\#detection''
(e.g. ``64738-3-6473''),
which documents the INT exposure number (\#run),
the CCD number (\#ccd),
and the row number in the CASU detection table (\#detection)
-- see columns \emph{rDetectionID},
\emph{iDetectionID}, \emph{haDetectionID}.

Photometry is provided based on the 2.3-arcsec diameter circular aperture
by default (columns \emph{r}, \emph{i}, \emph{ha}).
The choice of this aperture size as the default 
is based on a trade-off between concerns 
about small number statistics and centroiding errors
for small apertures on one hand,
and diminishing signal-to-noise ratios and source confusion
for large apertures on the other hand.
The user is not restricted to this choice because
the catalogue also provides magnitudes
using three alternative aperture sizes:
the peak pixel height 
(columns \emph{rPeakMag}, \emph{iPeakMag}, \emph{haPeakMag}),
a circular 1.2-arcsec-diameter aperture 
(\emph{rAperMag1}, \emph{iAperMag1},
 \emph{haAperMag1}) and
a 3.3-arcsec-diameter aperture 
(\emph{rAperMag3}, \emph{iAperMag3},
 \emph{haAperMag3}).

Each of these magnitude measurements have been
corrected for the flux lost outside of their respective apertures,
using a correction term which is inferred from the
mean shape of the PSF measured locally in the CCD frame.
In the case of a point source,
the four alternative magnitudes are expected
to be consistent with each other
within the photon noise uncertainties
(which are given in columns \emph{rErr}, \emph{rPeakMagErr},
\emph{rAperMag1Err}, \emph{rAperMag3Err}, etc).
When this is not the case,
it is likely that the source is either
an extended object or that it has
been incorrectly measured as a result of
source blending or a rapidly spatially-varying nebulous background.
In \S\ref{sec:qualitycriteria} we will explain that the consistency
of the magnitude measurements in the different apertures
can be used as a criterion for selecting stellar objects
with reliable photometry from the catalogue.

The brightness of each object as a function of increasing
aperture size is also used by the CASU pipeline to provide
a discrete star/galaxy/noise classification flag
(\emph{rClass}, \emph{iClass}, \emph{haClass})
and a continuous stellarness-of-profile statistic
(\emph{rClassStat}, \emph{iClassStat}, \emph{haClassStat}).
For convenience, we have combined
these single-band morphological measures
into band-merged class probabilities and flags
(\emph{pStar}, \emph{pGalaxy}, \emph{pNoise},
\emph{mergedClass}, \emph{mergedClassStat})
using the merging scheme in use at the WSA\footnote{Explained at
http://surveys.roe.ac.uk/wsa/www/gloss\_m.html \#gpssource\_mergedclass
}.


Information on the quality of each detection is included 
in a series of additional columns.
We draw attention to three useful flags
which warn about the likely presence of systematics:
\begin{enumerate}
\item The \emph{saturated} column is used to flag sources
for which the peak pixel height exceeds 55000 counts,
which is typically the case for stars brighter than 12-13th magnitude.
Although the pipeline attempts to extrapolate the brightness of
saturated stars based on the shape of their PSF,
such extrapolation is prone to errors
and we do not recommend their use.
\item The \emph{deblend} column is used to flag sources 
which partially overlap with a nearby neighbour.
Although the pipeline applies a deblending procedure
to such objects, the procedure is currently applied separately
in each band and hence the $r$-$i$ and $r$-\ha\ colours
of such objects are prone to errors.
\item The \emph{brightNeighb} column is used to flag sources which are located
within 5 arcmin from an object brighter than $V=7$ 
according to the Bright Star Catalogue (BSC; Hoffleit et al. 1991), 
or within 10 arcmin if the neighbour is brighter than $V=4$.
Such very bright stars are known to cause systematic errors
and spurious detections as a result of stray light 
and diffraction spikes.
\end{enumerate}
In addition to the above, we also created warning flags for internal bookkeeping.
For example, we flagged detections which fell in the strongly vignetted regions of the focal plane,
which were truncated by CCD edges,
or which were otherwise affected by bad pixels in the detector.
We will explain below that none of such detections 
have been included in the catalogue
-- an alternative detection was available in essentially all these situations
because of the IPHAS field pair strategy --
and hence these internal warning flags do not appear
in the final source catalogue.

Finally, we note that basic information on the observing conditions
is included (\emph{fieldID}, \emph{fieldGrade}, \emph{night}, \emph{seeing}).
A table containing more detailed quality control information,
indexed by \emph{fieldID}, is made available on our website.

\subsection{Band-merging the detection tables}

The second step in compiling the source catalogue
is to merge the contemporaneous trios
of $r$, $i$, \ha\ detection tables
into multi-band field catalogues.
This required a positional matching procedure 
to link sources between the three bands
based on their position on the sky.
We used the \textsc{tmatchn} function 
of the \textsc{stilts} software for this purpose,
which allows rows from multiple tables to be matched \citep{Taylor2006}.
The result of the procedure is a band-merged catalogue
in which each row corresponds to a group of linked $r$/$i$/\ha\ detections
which satisfy a maximum matching distance criterion in a pair-wise sense.
Sources for which no counterpart was identified
are retained in the catalogue as single-band detections.

We employed a maximum matching distance of 1~arcsec,
which was chosen based on a trade-off between 
completeness and reliability.
On one hand, a matching distance larger than 1~arcsec 
was found to allow too many spurious and unrelated sources 
to be linked. 
On the other hand, a value smaller than 1~arcsec 
would pose problems for very faint sources 
with large centroiding errors, 
and would occasionally fail to link detections near CCD corners
where the astrometric solution can 
show systematic errors which exceed 0.5~arcsec.
The position offsets between the $r$ and $i$/\ha\ detections
have been included in the catalogue 
and can hence be tightened by the user if necessary
(columns \emph{iXi}, \emph{iEta}, \emph{haXi}, \emph{haEta}).
We note that also UKIDSS/GPS adopted 
a maximum matching distance of 1 arcsecond 
for similar reasons \citep{Hambly2008}.

The resulting band-merged catalogues were found
to be reliable for the vast majority of fields.
We warn that source blending and confusion is unavoidable
for faint objects in the Galactic Plane however;
in \S\ref{sec:discussion} we will show
that 19\% of the sources in our catalogue
are flagged as blended objects (column \emph{deblend})
and their band-merged data should be treated with care
because they may have fallen victim to source confusion
during the band-merging step.

\subsection{Selecting the primary detections}

We explained earlier that the survey contains repeat observations
of identical sources as a result of overlaps in the data-taking pattern.
Amongst all sources in the reliable magnitude range $13<r<19$,
we find that 65\% were detected twice and 25\% were detected three times or more.
Only 9\% were detected once.
Unsurprisingly, their spatial distribution traces
the inter-CCD gaps and footprint edges.

The principal aim of this data release is to provide 
reliable photometry at a single epoch,
and hence we have decided
to focus on providing the magnitudes
and coordinates using only the \emph{best-available} 
detection of each object -- 
hereafter referred to as the \emph{primary} detection.
Although overlapping fields could have been co-added 
to gain a small improvement in depth, 
we have decided against this for two reasons.
Firstly, combining the information from multiple epochs
would make the photometry of variable stars difficult to interpret.
Secondly, co-adding would cause the image quality to degrade towards the mean,
which is a draw-back for crowded fields.

Anyone interested in the alternative detections of a source
-- hereafter called the \emph{secondary} detections --
can nevertheless obtain this information in two ways.
To begin with, whenever a secondary detection was observed 
within 10 minutes of the primary,
the magnitudes of that secondary detection
have been included in the catalogue
(columns \emph{sourceID2}, \emph{fieldID2}, 
\emph{r2}, \emph{i2}, \emph{ha2},
\emph{rErr2}, \emph{iErr2}, \emph{haErr2}, \emph{errBits2}.
This is the case for 66\% of the sources brighter than $r<20$
due to the IPHAS field pair observing pattern.
In addition, the complete set of detection tables -- one for each exposure -- 
is made accessibly on our website to allow other uses of the data.
A user-friendly catalogue of secondary detections 
has not been compiled at present 
but may be part of a future data release.

The primary detection is defined as the entry in the 
set of band-merged field catalogues which provides 
the most reliable information for a unique source.
Primary detections have been selected using a so-called \emph{seaming} procedure
which has been adapted from the algorithm developed for the WSA\footnote{http://surveys.roe.ac.uk/wsa/dboverview.html\#merge}.
In brief, the first step of the procedure is to identify all the duplicate detections
by cross-matching the overlap regions of all field catalogues,
again using a maximum matching distance of 1\arcsec.
The duplicate detections for each unique source are then ranked according to
to (i) filter coverage, (ii) quality score (column \emph{errBits}),
and (iii) the average seeing of stars in the CCD frame rounded to 0.2~arcsec.
If this ranking scheme reveals multiple `winners' of identical quality,
then the one that was observed closest to the optical axis of the camera is chosen.


\subsection{Compiling the final source catalogue}

As the final step, the primary detections that have been
selected above are compiled
into the 98-column source catalogue
that is described in Appendix~\ref{app:columns}
and made available on line.
The entire list of sources naturally includes 
a significant number of spurious entries
as a result of the very sensitive detection levels
that are employed by the CASU pipeline by default.
To limit the size of the source catalogue,
we have decided to enforce three basic criteria
which must be met for a candidate source
to be included in the catalogue:
\begin{enumerate}
\item the source must have been detected at SNR$>5$ in at least
one of the bands, i.e. it is required that at least one of
\emph{rErr}, \emph{iErr} or \emph{haErr} is smaller
than 0.2 mag;
\item the shape of the source must not be an obvious
cosmic ray or noise artefact, i.e. we require
either \emph{pStar} or \emph{pGalaxy} to be
greater than 20\%;
\item the source must not have been detected in one of the strongly
vignetted corners of the detector, 
not have had any known bad pixels in the aperture,
and not have been on the edge of one of the CCDs,
i.e. we require the \emph{errBits} quality score
to be smaller than 64.
\end{enumerate}

A total of 219 million primary detections satisfied
the above criteria and have been included in the catalogue.
Amongst these, 158 million objects 
are detected in all three bands (72\%),
30 million are detected in two bands (14\%),
and 31 million entries are single-band detections (14\%).
Roughly half of the single-band detections were made in the $i$-band.
This is likely explained by the fact that the $i$-band is least
affected by interstellar extinction and can occasionally pick up
highly-reddened objects which are otherwise lost in $r$/\ha.

\section{Discussion}
\label{sec:discussion}

Having explained how the catalogue was generated,
we now offer an overview of its properties
by discussing  
(i) the recommended quality criteria,
(ii) the typical photometric uncertainties,
and (iii) the source densities and the frequency of source blending.

\subsection{Recommended quality criteria}
\label{sec:qualitycriteria}

Like any other photometric survey,
the majority of the objects in our catalogue
are faint sources observed near the detection limits:
55\% of the entries in the catalogue
are fainter than $r > 20$
and 18\% are even fainter than $r > 21$.
The measurements of such faint objects
are naturally prone to large
random and systematic uncertainties,
for example, an inaccurately subtracted background
will introduce a proportionally larger systematic error
in a faint object.
Most scientific applications will require a set of
quality criteria to be enforced for the purpose
of removing faint and low-quality objects.

The choice of quality criteria is often a complicated
trade-off between completeness on one hand
and accuracy on the other.
To aid users we have listed two sets of
recommended quality criteria 
in Tables~\ref{tab:reliable} and \ref{tab:veryreliable}.

Firstly, Table~\ref{tab:reliable} details
a set of minimum quality criteria
which should benefit most applications
which require reliable $r-i$ and $r-$\ha\ colours
without removing more than $\sim$80\%
of the sources brighter than $r < 19$.
The listed criteria are designed to 
(i) remove low-SNR sources, 
(ii) remove saturated sources,
and (iii) remove objects for which the 2.3~arcsec diameter
aperture magnitudes are inconsistent 
with the alternative 1.2~arcsec diameter aperture measurements.
The last criterion is a proxy
for identifying objects which are affected
by inaccurate background subtraction
or failed source deblending.
A total of 86 out of 219 million sources 
(39\%) satisfy all the criteria listed in Table~\ref{tab:reliable}
and are hereafter referred to as ``reliable''.
For convenience, the catalogue contains a boolean column
named \emph{reliable} which flags these objects
and makes their selection easy.

\begin{table*}
\vspace{2cm}

\begin{tabular}{p{8cm}lp{6cm}}
  \hline
  Quality criterion & Rows passed & Description \\
  \hline
  rErr\,$<$\,0.1 {\sc and} iErr\,$<$\,0.1 {\sc and} haErr\,$<$\,0.1 &
  109 million (50\%) &
  Require the photon noise to be less than 
  0.1 mag in all bands (i.e. SNR$>$10).
  This implicitly requires a detection in all three bands.  \\
  $r>13$ {\sc and} $i>12$ {\sc and} \ha\,$>12.5$ {\sc and not} \emph{saturated} &
  158 million (72\%) &
  The brightness must not exceed the nominal saturation limit
  and the peak pixel height must not exceed 55\,000 counts.
  Again, this implicitly requires a detection in all three bands.
  \\
  $|\mathrm{r}- \mathrm{rAperMag1}| 
  < 3\sqrt{\mathrm{rErr}^2 + \mathrm{rAperMag1Err}^2} 
  + 0.03$ &
  176 million (80\%) &
  Require the $r$ magnitude measured 
  in the default 2.3\arcsec\ diameter aperture
  to be consistent with the measurement 
  made in the smaller 1.2\arcsec\ aperture,
  albeit tolerating a 0.03 mag systematic error.
  This will reject sources for which the background
  subtraction or the deblending procedure
  was not performed reliably. \\
  $|\mathrm{i}- \mathrm{iAperMag1}| 
  < 3\sqrt{\mathrm{iErr}^2 + \mathrm{iAperMag1Err}^2} 
  + 0.03$ &
  183 million (84\%) &
  Same as above for $i$. \\
  $|\mathrm{ha}- \mathrm{haAperMag1}| < 
  3\sqrt{\mathrm{haErr}^2 + \mathrm{haAperMag1Err}^2} 
  + 0.03$ &
  158 million (72\%) &
  Same as above for \ha. \\
  \hline
  All of the above (flagged as {\bf\emph{reliable}}) &
  86 million (39\%) & \\
  \hline
\end{tabular}
\caption{Recommended minimum quality criteria 
for selecting objects with reliable colours 
from the IPHAS DR2 source catalogue. 
86~million entries in the catalogue (39\%)
satisfy all the criteria listed in this table.
For convenience, these have been flagged in the catalogue
using the column named \emph{reliable}.}
\label{tab:reliable}

\vspace{2cm}

\begin{tabular}{p{8cm}lp{6cm}}
  \hline
  Quality criterion & Rows passed & Description \\
  \hline
   reliable &
   86 million (39\%) &
   The object must satisfy the criteria listed in Table~\ref{tab:reliable}. \\
   
   pStar $>$ 0.9 &
   145 million (66\%) &
   % pStar > 0.89 is identical to mergedClass == -1
   The object must appears as a perfect point source,
   as inferred from comparing its Point Spread Function (PSF)
   with the average PSF measured in the same CCD. \\
   
   {\sc not} \emph{deblend} &
   177 million (81\%) &
   The source must appear as a single, unconfused object. \\
   
   {\sc not} \emph{brightNeighb} &
   216 million (99\%) &
   There is no star brighter than $V < 4$ within 10 arcmin, 
   or brighter than $V < 7$ within 5 arcmin.
   Such very bright stars cause scattered light and diffraction spikes,
   which may add systematic errors to the photometry
   or even trigger spurious detections. \\  
  \hline
  
  All of the above (flagged as {\bf\emph{veryReliable}}) &
  59 million (27\%) & \\
  \hline
\end{tabular}
\caption{Additional quality criteria which are recommended
for applications which require very reliable colours
at the expense of completeness. 
For convenience, the sources which satisfy the criteria listed
in this table have been flagged in the catalogue
using the column named \emph{veryReliable}.}
\label{tab:veryreliable}

\vspace{2cm}
\end{table*}

For applications which require
an even higher standard of reliability,
a further set of additional quality criteria
are listed in Table~\ref{tab:veryreliable}.
These criteria are designed to ensure that
(i) the object appeared as a perfect point source,
(ii) the object was not blended with a nearby neighbour,
and (ii) the object was not located near a very bright star.
59 million sources (27\%) satisfy
these additional criteria 
and are hereafter referred to as ``very reliable''.
Again, the catalogue contains a boolean column
named \emph{veryReliable} which flags these objects.

Figure~\ref{fig:magdist} compares the $r$-band magnitude
distribution for objects with and without the 
\emph{reliable} and \emph{veryReliable} criteria applied. 
We find that 81\% of the sources 
in the magnitude range $13 < r < 19$
are considered \emph{reliable},
which drops to 72\% in the range $19 < r < 20$
and 9\% at $r>20$.
In contrast, only 54\% of the sources 
in the magnitude range $13 < r < 19$
are considered \emph{veryReliable}.
The stricter criteria filter out a lot
of objects at early Galactic longitudes 
where source blending is a common problem
(we will demonstrate this in \S\ref{sec:densities}).
The \emph{veryReliable} flag should hence
only be used in applications which require very reliable photometry
at the expense of completeness,
which might be the case for e.g. spectroscopic target selection.

\begin{figure}
    \includegraphics[width=0.5\textwidth]{./plots/magdist/magdist-r.pdf} 
    \caption{r-band magnitude distribution for all objects in the catalogue 
    (light grey), for objects flagged as \emph{reliable} 
    according to the criteria set out in Table~\ref{tab:reliable} (grey),
    and for objects flagged as \emph{veryReliable} 
    following Table~\ref{tab:veryreliable} (dark grey).
    The magnitude distributions for $i$ and \ha\
    look identical, apart from being shifted
    by about 1 and 0.5 mag towards brighter magnitudes,
    respectively.}
    \label{fig:magdist}
\end{figure}

It is easy to see how the quality criteria
may be adapted to be more tolerant.
For example, by raising the allowed photometric uncertainties
from 0.1 mag to 0.2 mag one can retrieve 42 million candidate sources.

\subsection{Photometric uncertainties}

Figure~\ref{fig:uncertainties} shows the mean photometric
uncertainties as a function of magnitude for each band.
We find that the uncertainties typically
reach 0.1 mag near 20.5 in $r$ 
and 19.5 in $i$/\ha when the default 2.3\arcsec\ aperture is used.
At this point we note that the average colour
of objects in the survey is
$1.06\pm0.12$ for ($r$-$i$) and $0.44\pm0.03$ for ($r$-\ha).
The better depth of $r$ is hence compensated
by the fact that stars tend to have 
brighter magnitudes in $i$ and \ha.

\begin{figure}
    \includegraphics[width=0.5\textwidth]{./plots/errors/uncertainties.pdf} 
    \caption{Mean photometric uncertainties
             for $r$ (top), $i$ (middle) and \ha\ (bottom).
             Data points shown are the average values of
             columns \emph{rErr}, \emph{iErr} and \emph{haErr}
             in the catalogue, 
             and the errorbars show the standard deviations.
             The dashed and solid lines indicate 
             the 10$\sigma$ and 5$\sigma$ limits, respectively.
             These uncertainties are based only on the (Poissonian)
             photon noise and hence this figure does not show
             systematic or calibration uncertainties.}
    \label{fig:uncertainties}
\end{figure}

The uncertainties shown in Fig.~\ref{fig:uncertainties}
are the random errors based on the expected Poissonian photon noise.
Systematics, such as calibration and deblending errors,
are not included.
To appraise the extent to which our photometry is affected
by such systematics, we can exploit the
secondary measurements which were made as part of the
field-pair observing strategy and are available for 51\%
of the sources in the catalogue.

Figure~\ref{fig:pairmag} shows the mean residuals between
the primary and secondary magnitudes
-- i.e. the average difference between catalogue columns \emph{r}-\emph{r2},
\emph{i}-\emph{i2}, \emph{ha}-\emph{ha2} -- as a function of magnitude.
We find that sources across the magnitude range 
$13 < r < 17$ are consistent at the level of 5\%
(i.e. $\sigma_{r-r2} \le 0.05$ mag),
with the best repeatability
reached at $r=14$ ($\sigma_{r-r2}~=~0.041$~mag).
We draw attention to the fact that brighter stars
tend to show significantly larger residuals
-- e.g. $\sigma_{r-r2}~=~0.14$~mag at $r=12$ --
which is due to saturation effects.
At the faint end we find residuals which
show significantly more scatter than would be expected
from photon noise alone, that is,
the effects of source blending and background subtraction
appear to dominate from $\sim$20th magnitude onwards.

\begin{figure*}
    \vspace{1cm}
    \includegraphics[width=0.5\textwidth]{./plots/pairmag/pairmag.pdf} 
    \caption{Photometric repeatability illustrated by plotting
             the mean residuals between all the primary and secondary detections
             in the catalogue as a function of magnitude.
             The best photometric repeatability is reached at $r=14$
             with $\sigma_{r-r2}~=~0.041$~mag.
             Note that bright stars at $r<13$ and $i<12$ 
             show increasing uncertainties due to saturation effects.}
    \label{fig:pairmag}
    \vspace{1cm}
    \includegraphics[width=0.5\textwidth]{./plots/pairmag/pairmag-reliable.pdf} 
    \caption{Same as Figure~\ref{fig:pairmag},
    except that only the subset of sources
    flagged as \emph{veryReliable} are now included.
    We find that applying the quality criteria
    has improved the photometric repeatability significantly.
    The best repeatability is again reached at $r=14$
    but has reduced to $\sigma=0.028$ mag.
    The quality criteria have also been successful
    at removing objects with large systematics at the bright
    and faint ends.}
    \label{fig:pairmag_reliable}
\end{figure*}

In Figure~\ref{fig:pairmag_reliable} we show 
a similar comparison of the primary and secondary detections,
but this time we have only include sources which are flagged
as \emph{veryReliable} in the catalogue
(i.e. not saturated, not confused, not near bright stars, etc.)
We find that the average residuals are significantly better
for this subset of the catalogue. Sources across the magnitude range 
$13 < r < 17$ are consistent at the level of 0.03 mag,
and the best repeatability is again reached at $r=14$
with $\sigma_{r-r2}~=~0.028$~mag.
We conclude that the \emph{veryReliable} quality criteria are effective
in reducing the systematic errors to
the same level as the accuracy of the
global photometric calibration (cf. \S\ref{sec:calibration}).
Moreover, the large systematics at the bright and faint end
have disappeared.

\subsection{Source densities and blending problems}
\label{sec:densities}

The mean source density as a function of Galactic longitude
is shown in Figure~\ref{fig:density} (thick blue line).
The densities were computed by counting the 
number of sources in $1^\circ$-wide longitude bins
across which covered the entire latitude range $-5^\circ<b<+5^\circ$.
Unsurprisingly, we find the average source densities to increase
towards the Galactic centre.
For example, the average source density near $l\simeq 30^\circ$
is roughly 30\,000 objects per square degree,
which is five times more than the density
found near $l\simeq 180^\circ$.

In addition to the global trend, 
there are significant variations in the source density on smaller scales.
For example,  we find a significant drop near the constellations 
of Cygnus ($l\simeq 80$) and Aquila ($l\simeq 40$),
which are regions known to be affected
by high levels of foreground extinction.
Dark clouds are visible towards these constellation by eye,
and they are often referred to as ``the Great Rift''.

We warn however that the densities reported here 
have not been corrected for survey completeness
or differences in the observing conditions across the survey.
For example, the dip in the density near $l\simeq140^\circ$
is an artificial feature caused by gaps
in the footprint coverage (which are apparent in Fig.~\ref{fig:footprint}).
In a forthcoming paper,
we aim to calibrate the IPHAS-based source densities
by injecting artificial stars into the IPHAS images
and measuring their recovery rate (Farnhill et al., in preparation).
Indeed IPHAS has the potential to offer
calibrated, two-dimensional stellar density maps
which can be used to constrain detailed models of our Galaxy,
but it is beyond the scope of the present work.

\begin{figure*}
    \includegraphics[width=\textwidth]{./plots/density/density.pdf} 
    \caption{Mean number density of sources in the catalogue 
    as a function of Galactic longitude, 
    with and without blended sources included. 
    The densities shown were computed by counting the sources 
    at each longitude between $-5^\circ<b<+5^\circ$ (upper blue line).
    We also show the densities based on only counting those sources 
    for which the \emph{deblend} flag is {\sc false}, 
    i.e. unconfused sources for which the CASU pipeline did not have to apply 
    a deblending procedure (lower red line). 
    }
    \label{fig:density}
\end{figure*}

In Figure~\ref{fig:density} we also shows the density
of non-blended sources (thin red line).
These are sources for which the \emph{deblend} flag is {\sc false},
i.e. sources for which the CASU pipeline did not have to apply 
a deblending procedure to separate the flux
originating from two or more overlapping objects.
We find a strong correlation between the source density
and the fraction of sources affected by source blending.
For example, only NN\% of the sources are blended
at $l>90^\circ$, whereas NN\% are blended at $l<90^\circ$.

As we explained earlier, blended sources must be used with caution.
Firstly, the deblending-procedure crucially depends
on the local PSF being measured accurately. 
Secondly, blended sources may are likely candidates to have fallen victim
to source confusion during the band-merging procedure.
In future work we hope to investigate the use of
more advanced PSF-fitting routines
in which sources are measured simultaneously across all bands,
perhaps guided by an external list of sources provided
by near-infrared surveys or Gaia data.

\section{Demonstration}
\label{sec:demonstration}

We conclude this paper by demonstrating how the unique
$r$-$i$/$r$-\ha\ colour-colour diagram offered by this catalogue
can readily be used to
(i) characterise the extinction regime at different sightlines, and
(ii) identify \ha\ emission-line objects.

\subsection{Colour-colour and colour-magnitude diagrams}

The survey's unique $r$-\ha\ colour,
when combined with $r$-$i$,
has been shown to provide simultaneous constraints 
on intrinsic stellar colour and interstellar extinction \citep{Drew2008}. 
That is, the main sequence in the $r$-$i$/$r$-\ha\ diagram
runs in a direction that is different from the reddening vector,
because the $r$-\ha\ colour tends to act as a coarse proxy for spectral type
and is less sensitive to reddening than $r$-$i$.
As a result, the distribution of a stellar population
in the IPHAS colour-colour diagram
can offer a handle on the properties and extinction regime
along a line of sight.

This is demonstrated in Figures \ref{fig:l180}, \ref{fig:l45} \& \ref{fig:l30},
where we present three sets of IPHAS colour/magnitude diagrams
towards three distinct sightlines
located at Galactic longitudes $180^\circ$, $45^\circ$ and $30^\circ$,
respectively.
Each figure contains all the sources which are flagged as \emph{veryReliable}
and are located in a region of one square degree 
centred on the coordinates indicated in the diagram
(i.e. within a radius of $0.564^\circ$ from the indicated sightline).
For clarity, we have imposed the additional criterion
for photometric uncertainties to be smaller than 0.05 mag in each band
(corresponding to a magnitude limit near $\sim$19th magnitude, effectively).

Each of the diagrams reveals a well-defined locus,
which demonstrates the health of the catalogue and the global calibration
for investigating stellar populations across wide areas.
We have annotated the colour-colour diagrams
by showing the position of the unreddened main sequence (thin solid line),
the unreddened giant branch (thick solid line)
and the reddening track for an A0V-type star (dashed line)
-- all three are based on the \cite{Pickles1998} library of empirical spectra
tabulated for IPHAS by \cite{Drew2005}.
In the colour-magnitude diagrams we only show the reddening vector
together with the unreddened 1~Gyr isochrone due to \cite{Bressan2012},
which are made available for the IPHAS filter system through a
popular on line tool hosted by the Observatory of Padova (http://stev.oapd.inaf.it/cmd).
The isochrone and reddening vector has been placed
at an arbitrary distance of 2~kpc.

Each of the sightlines reveals a stellar population
with distinct characteristics.
Firstly, towards the Galactic anti-centre at $l=180^\circ$ (Fig.~\ref{fig:l180}) 
we find a population dominated by lowly-reddened main sequence stars.
This is consistent with the estimated total sightline extinction of $E(B-V)=0.49$
given by \cite{Schlegel1998}, following the correction of \cite{Schlafly2011}.
Looking in more detail we can see that the stellar locus is narrower for M-type dwarfs than for earlier types;
we do not observe M dwarfs expereincing the strongest reddening possible for this sightline.
This implies that extinction is still increasing at distances beyond $\sim 2$~kpc,
where M dwarfs are too faint to be contained in the IPHAS catalogue.
It is also clear that there are no unreddened stars earlier than $\sim$K0 visible,
such stars would be saturated if within a few hundred parsecs.
This therefore suggests that there is a measurable increase in extinction locally.
We also note an relative absence of late type giants which,
due to the relative brevity of the corresponding phase of stellar evolution,
would only account for a small proportion of a volume limited sample.

 
In contrast, closer towards the galactic
centre at $l=45^\circ$ (Fig.~\ref{fig:l45})
we find a wealth of reddened objects
which appear to be separated into a lowly
and a highly reddened component,
perhaps revealing two distinct parts of the Galaxy.

Finally, in one of our earliest sightlines at $l=30^\circ$
we find a very high number of extremely reddened giants
in addition to an unreddened population
of foreground dwarfs.

The number density of stars in the colour-colour
and colour-magnitude space
can be compared against population synthesis models
to create three-dimensional maps of the extinction
across several kpc \citep{Sale2009,Sale2012}.
Such an extinction map based on our catalogue
is to be released in a separate paper
that accompanies this data release (Sale et al., in preparation).

\begin{figure*}
    \begin{minipage}[b]{\linewidth}
        \includegraphics[width=0.5\textwidth]{./plots/ccd-180-3.pdf} 
        \includegraphics[width=0.5\textwidth]{./plots/cmd-180-3.pdf}
    \end{minipage}
    \caption{Colour-colour and colour-magnitude diagram (left and right panel)
    showing sources flagged as \emph{veryReliable}
    located in an area of one square degree
    centred near the Galactic anti-centre 
    at $(l,b)=(180^\circ,+3^\circ)$.
    The colour-colour diagram shows the
    position of the main sequence (thin solid line),
    giant stars (thick solid line)
    and the reddening track for an A0V-type star (dashed line)
    based on the \citet{Pickles1998} library of empirical spectra.
    The colour-magnitude only show the reddening vector
    along with the unreddened 1~Gyr isochrone due to \citet{Bressan2012},
    which has been placed at an arbitrary distance of 2~kpc for reference.
    This is one of the least reddened sightlines
    in the survey %(Av XX, Schlegel et al. XXXX)
    and hence the observed stellar population appears to be dominated 
    by lowly reddened main sequence stars.}
    \label{fig:l180}
    \begin{minipage}[b]{\linewidth}
        \includegraphics[width=0.5\textwidth]{./plots/ccd-45-2.pdf}
        \includegraphics[width=0.5\textwidth]{./plots/cmd-45-2.pdf}
    \end{minipage}
    \caption{Same as above for $(l,b)=(45^\circ,+2^\circ)$,
    which is one of the highest-density sightlines in the survey,
    revealing a population of stars with a reddening distribution
    that appears to be bi-model.}
    \label{fig:l45}
    \begin{minipage}[b]{\linewidth}
        \includegraphics[width=0.5\textwidth]{./plots/ccd-30-0.pdf}
        \includegraphics[width=0.5\textwidth]{./plots/cmd-30-0.pdf} 
    \end{minipage}
    \caption{Same as above for $(l,b)=(30^\circ,0^\circ)$.
    This is one of the most reddened sightlines in the survey.
    %(XXX, Schlegel et al XXXX).
    }
    \label{fig:l30}
\end{figure*}

\subsection{Identifying \ha\ emission-line objects}

A primary motivation for carrying
out the survey 
was to enable the discovery of 
new emission-line objects across the Galactic Plane.
\ha\ in emission is a well-known tracer
for stars in the short-lived pre- or
post-main sequence stages of their evolution,
and hence IPHAS aims to allow larger, deeper
and more statistically robust samples of such rare objects
to be established.
The survey-wide identification and analysis 
of such stars is beyond the scope of the present work,
but in this section we demonstrate how the
catalogue may be used for this purpose.

An initial list of candidate \ha-emitters
based on the first IPHAS data release has previously
been presented by \cite{Witham2008}. 
Because no global uniform calibration was available
at the time, \citeauthor{Witham2008} employed 
a sigma-clipping technique to select objects with
large, outlying $r$-\ha\ colours.
In contrast, the new catalogue
allows objects to be picked out
from the $r$-$i$/$r$-\ha\ colour-colour diagram
using model-based colour criteria
rather than a statistical procedure.
In what follows we demonstrate this ability 
by selecting candidate emission-line objects
towards a small region in the sky.

The target of our demonstration is Sh 2-82:
a 5~arcmin-wide H{\sc ii} region located near $(l,b)=(53.55^\circ, 0.00^\circ)$
in the constellation of Sagitta.
Nicknamed by amateur astronomers as the ``Little Cocoon Nebula'',
Sh 2-82 is ionised by 
the $\sim$10th magnitude star HD\,231616
% Georgelin1973 claims B0III, others claim B0V
with spectral type B0V/III
\citep{Georgelin1973,Mayer1973,Hunter1990}.
The ionising star has been placed at a likely distance of 1.5-1.7 kpc
based on the photometric parallax
\citep{Mayer1973,Lahulla1985,Hunter1990}.

Figure~\ref{fig:mosaic_iphas} shows a 20-by-15 arcmin
colour mosaic centred on Sh 2-82,
composed of our \ha\ (red channel),
$r$ (green channel),
and $i$ (blue channel) images.
The ionising star can be seen as the bright object
near the centre of the H{\sc ii} region,
which is surrounded by a faint reflection nebula
and several dark cloud filaments.
For comparison, Figure~\ref{fig:mosaic_spitzer} shows
a mosaic of the identical region 
as seen by the Spitzer Space Telescope
in the mid-infrared. The Spitzer image
reveals a bubble-shaped structure of warm dust
which surrounds the entire H{\sc ii} region.

\begin{figure*}
    \begin{minipage}[b]{0.8\linewidth}
        \includegraphics[width=\textwidth]{./plots/mosaic/sh2-82-iphas.pdf} 
    \end{minipage}
\caption{IPHAS-based mosaic of H{\sc ii} region Sh 2-82,
composed of \ha\ (red channel), $r$ (green channel) and $i$ (blue channel). Yellow triangles show the position of candidate \ha-emitters
which have been selected from the colour-colour diagram
in Figure~\ref{fig:emitters}. Note that the H{\sc ii} region is surrounded by a faint blue/green reflection nebula
and dark cloud filaments.}
\label{fig:mosaic_iphas}
    \begin{minipage}[b]{0.8\linewidth}
        \includegraphics[width=\textwidth]{./plots/mosaic/sh2-82-spitzer.pdf} 
    \end{minipage}
    \caption{Star-forming region Sh 2-82 as seen in the mid-infrared
    by the Spitzer Space Telescope. The mosaic is composed of the 24\,\micron\ (red), 8.0\,\micron\ (green) and 4.5\,\micron\ (blue) bands.
    The image reveals a bubble-shaped structure which surrounds the {\sc Hii} region that is seen in the IPHAS mosaic of the same region (Figure~\ref{fig:mosaic_iphas}).}
    \label{fig:mosaic_spitzer}
\end{figure*}

Figure~\ref{fig:emitters} presents
the IPHAS colour-colour diagram for the region covered by the
mosaic images.
Gray circles show all objects in the region
which are brighter than $r<20$
and have been flagged as \emph{reliable}
in our catalogue.
The diagram also shows the unreddened main sequence (solid line)
and the expected position of unredded main-sequence stars
with \ha\ in emission at a strength of EW=$-10{\rm \AA}$ (dashed line),
taken from the colour simulations due to \citet{Barentsen2011a}.
Six stars are found to lie confidently above the 
dashed line at the level of $3\sigma$ 
(i.e. the distance between the dashed line is larger than
three times the uncertainty on the $r$-\ha\ colour).
These reliable candidate \ha-emitters
are marked by red triangles in the colour-colour diagram
and their photometry is detailed in Appendix~\ref{app:emitters}.

\begin{figure}
  \includegraphics[width=0.45\textwidth]{./plots/sh2-82-ccd.pdf}
    \caption{$r$-$i$/$r$-\ha\ diagram for the rectangular region of 
    20-by-15 arcmin centred on the H{\sc ii} region Sh 2-82,
    which is the area shown in Figure~\ref{fig:mosaic_iphas}.    
    The diagram shows all objects in the catalogue
    which have been flagged as \emph{reliable} and are brighter
    than $r<20$ (grey circles).
    The unreddened main sequence is indicated by a solid line,
    while the main sequence for stars with an \ha\ emission line
    strength of $-10\,\rm{\AA}$ EW is indicated by a dashed line,
    following the colour simulations due to \citet{Barentsen2011a}.
    Red triangles indicate objects which have been identified as
    as highly likely \ha-emitters (see text).}
    \label{fig:emitters}
\end{figure}

The spatial distribution of our six candidate emission-line objects
is marked by yellow triangles in the colour mosaic (Fig.~\ref{fig:mosaic_iphas}).
They are likely to be genuine young stars for two reasons.
Firstly, two of our candidates have recently been identified as likely
candidate Young Stellar Objects (YSO)
in an investigation of the region by \cite{Yu2012}.
In their study, the authors used 2MASS and Spitzer data
to select likely young stars by selecting objects with
circumstellar disks based on the infrared colour excess.
Secondly, we find that the four remaining objects are 
also detected in the Spitzer-based image,
although their colours are less extreme
than those identified as likely YSOs by \citeauthor{Yu2012}.

Prior to IPHAS, this region was essentially unstudied
at faint magnitudes in visible light.
\cite{Lahulla1985} reported magnitudes for 8 stars in the optical at 
$V < 15$. In contrast, the IPHAS catalogue offers photometry
for NN stars in the region down to $r<20$.
This demonstrates the ability of IPHAS to providing complimentary
for the wealth of poorly-studied star-forming regions
at low Galactic latitudes,
which have been unveiled in recent years
by the observed wealth of star-forming ``bubbles''
at mid-infrared wave%\cite{Bica2003} was the first to identify a
%near-infrared cluster
%associated with this region using near-infrared 2MASS data,
%and only recently has this cluster been confirmed
%in a short investigation which combined 2MASS
%and Spitzer data \cite{Yu2012}.
%Many of the young stars identified by \citeauthor{Yu2012}
%can be seen as pink and red stars in Fig.~\ref{fig:mosaic_spitzer}.lengths \cite[e.g.][]{Churchwell2006,Simpson2012}.
%In fact the cluster of young stars associated with Sh 2-82
%was only first reported by \cite{Bica2003},
%remained unstudied until the study by \cite{Yu2012}.

%Amongst the six emitters, N are detected at SNR X
%in the Spitzer X micron band. 

%This demonstrates how IPHAS is providing complimentary
%data for some of the thousands of star-forming ``bubbles'' have been unveiled at mid-infrared wavelengths \cite[e.g.][]{Churchwell2006,Simpson2012}.
%Optically-unveiled members of such star-forming regions
%may help to constrain extinctions and distances towards such regions \citep{Barentsen2013}.
%Moreover, in \cite{Barentsen2011a} we showed
%how the history of star formation in a region,
%and the possible triggered nature, may be discussed.
%This is particularly true 

%TODO: annotate -10 EW text on plot?
%TODO: mention Gaia, e.g. IPHAS for transient identification?

%\subsection{Caveats and lessons learnt}
%\cite{Bica2003} was the first to identify a
%near-infrared cluster
%associated with this region using near-infrared 2MASS data,
%and only recently has this cluster been confirmed
%in a short investigation which combined 2MASS
%and Spitzer data \cite{Yu2012}.
%Many of the young stars identified by \citeauthor{Yu2012}
%can be seen as pink and red stars in Fig.~\ref{fig:mosaic_spitzer}.

\section{Conclusions and future work}
\label{sec:conclusions}

A new data release for the IPHAS survey was presented,
taking the coverage up to over 90\% of the Northern Galactic Plane 
at $|b|<5^\circ$
and providing a uniform photometric calibration
for the first time.
We explained the data reduction and quality control procedures that
were applied, described and tested the new global photometric calibration,
and detailed the construction of the user-friendly source catalogue.

The observations included in this release
were found to achieve a median seeing of 1.1 arcsec
and a $5\sigma-$depth of $r=21.2\pm 0.5$, $i=20.0\pm 0.3$, \ha$=20.3\pm 0.3$.
The global calibration and photometric repeatability
is accurate at the level of $\sigma=0.03$ mag,
providing a significant improvement over the 
previous data release.
The source catalogue provides the best-available
single-epoch astrometry and photometry
for 219~million unique sources.

The data-taking strategy developed for IPHAS
have since been reapplied to carry out a companion INT/WFC survey called UVEX
in U/$g$/$r$ \citep{Groot2009},
and also southern counterpart to IPHAS and UVEX 
is being carried out in $u$/$g$/$r$/$i$/\ha\ 
called VPHAS+ (Drew et al, in press).
We hope to re-use the experience gained by this data release
to create similar releases for these companion surveys.

In future work, we aim to draw upon the PanSTARRS photometric
survey to further improve the accuracy of our global calibration.
We will also aim to correct the photometry for the
radial field distortions.
%We may also produce a catalogue which details all detections.

%TODO: mention Gaia?

\section*{Data access and source code}
\label{sec:dataaccess}

The catalogue is made available through the Vizier
catalogue tool (http://vizier.u-strasbg.fr),
where it is known as the ``IPHAS DR2 Source Catalogue''
(catalogue ID ???).
The full catalogue can also be downloaded in its entirety
from the IPHAS website (www.iphas.org) as a collection 
of binary FITS tables which comprise 50\,GB,
which is accompanied by a script
to ingest the data into a PostgreSQL database.
Our website also provides access to the pipeline-processed
imaging data, which we have updated to include
the re-calibrated DR2 zeropoint in the image headers.

The source code that was used to generate
the catalogue is available at
https://github.com/barentsen/iphas-dr2

\section*{Acknowledgments}

The IPHAS survey was carried out 
at the Isaac Newton Telescope (INT).
The INT is operated on the island of La Palma
by the Isaac Newton Group
in the Spanish Observatorio del Roque de los Muchachos
of the Instituto de Astrofisica de Canarias.
All data were processed 
by the Cambridge Astronomical Survey Unit,
at the Institute of Astronomy in Cambridge.

Preparation of the catalogue was eased greatly
by a number of software packages,
including the Python modules
AstroPy \citep{Astropy},
APLpy, NumPy and SciPy,
the PostgreSQL database software,
the TOPCAT and STILTS packages \citep{Taylor2005,Taylor2006},
and the Montage software maintained by NASA/IPAC.
We also made use of the SIMBAD, Vizier and Aladin \citep{Aladin} tools operated at CDS, Strasbourg, France.

This research made extensive use of
several complementary photometric surveys.
Our global calibration was aided
by the AAVSO Photometric All-Sky Survey (APASS),
funded by the Robert Martin Ayers Sciences Fund.
The calibration was tested against the
Sloan Digitized Sky Survey (SDSS),
funded by the Alfred P. Sloan Foundation, the Participating Institutions, the National Science Foundation, the U.S. Department of Energy, the National Aeronautics and Space Administration, the Japanese Monbukagakusho, the Max Planck Society, and the Higher Education Funding Council for England.
The astrometric pipeline reduction made
significant use of the Two Micron All Sky Survey (2MASS),
which is a joint project 
of the University of Massachusetts
and the Infrared Processing and Analysis Center/
California Institute of Technology,
funded by NASA and the NSF.

GB and JED acknowledge the support of a grant
from the Science \& Technology Facilities Council
of the UK (STFC, ref ST/J001335/1).
HJF is in receipt of an STFC postgraduate studentship.

\bibliographystyle{mn2e}
\bibliography{dr2}

\appendix
\input{app_columns.tex}

\section{Candidate emission-line objects towards Sh 2-82}
\label{app:emitters}
\begin{table*}
    \begin{tabular}{lccc}
    \toprule
    Name & $r$ & $i$ & \ha  \\
    \midrule
IPHAS2 J192954.40+181026.1& $17.69\pm0.01$ & $16.12\pm0.01$ & $16.19\pm0.01$ \\
IPHAS2 J193011.01+182051.2& $18.55\pm0.02$ & $16.95\pm0.02$ & $17.31\pm0.02$ \\
IPHAS2 J193021.52+181954.5& $19.72\pm0.05$ & $17.94\pm0.03$ & $18.47\pm0.04$ \\
IPHAS2 J193024.45+181938.3& $19.31\pm0.04$ & $17.55\pm0.02$ & $17.99\pm0.03$ \\
IPHAS2 J193033.00+181609.3& $18.25\pm0.01$ & $16.91\pm0.01$ & $16.92\pm0.01$ \\
IPHAS2 J193042.48+182317.4& $19.96\pm0.03$ & $18.11\pm0.03$ & $18.48\pm0.03$ \\
    \bottomrule
    \end{tabular}
    \caption{Candidate \ha-emitters towards Sh 2-82.}
    \label{tbl:emitters}
\end{table*}

\label{lastpage}

\end{document}\documentclass[useAMS,usenatbib]{mn2e}
\usepackage{amssymb,amsmath}
\usepackage[pdftex]{graphicx}
\usepackage{longtable} 
%\usepackage{lscape}
\usepackage{float}
\usepackage{booktabs}
\usepackage{aas_macros}

\def\ha{\mbox{H$\rm \alpha$}}
\def\arcsec{$''$}
\def\arcmin{$'$}
\def\deg{$^{\circ}$}
\def\micron{\mbox{$\mu$m}}

\title[IPHAS Data Release 2]{The Second Data Release 
of the INT Photometric H$\alpha$ Survey 
of the Northern Galactic Plane (IPHAS DR2)}
		
\author[G. Barentsen
et. al]{G. Barentsen$^{1}$\thanks{E-mail:geert@barentsen.be},
H. J. Farnhill$^{1}$,
J. E. Drew$^{1}$,
M. J. Irwin$^{4}$, \newauthor
E. A. Gonz$\acute{\rm{a}}$lez-Solares$^{4}$,
R. Greimel$^{2}$,
B. Mizalski$^{3}$,
C. Ruhland$^{1}$, \newauthor plus many friends.
\newauthor\\
$^{1}$Centre for Astrophysics Research, STRI, University of Hertfordshire, College Lane Campus, Hatfield AL10 9AB\\
$^{2}$Institute for Geophysics, Astrophysics and Meteorology, Institute of Physics, Karl-Franzens-Universit$\ddot{a}$t Graz, Universit$\ddot{a}$tsplatz 5, 8010 Graz, Austria\\
$^{3}$Building, Institute, Street Address, City, Code, Country\\
$^{4}$Institute of Astronomy, Madingley Road, Cambridge CB3 0HA}

\begin{document}
\date{Current draft typeset \today}
\pagerange{\pageref{firstpage}--\pageref{lastpage}} \pubyear{2013}

\maketitle

\label{firstpage}

\begin{abstract} % BACKGROUND, OBJECTIVE, METHODS, RESULTS, CONCLUSIONS
The INT/WFC Photometric H$\alpha$ Survey 
of the Northern Galactic Plane (IPHAS)
is a 1800 deg$^2$ imaging survey
covering the entire northern Milky Way at $|b| < 5^\circ$
in the $r$, $i$ and \ha\ filters 
using the Wide Field Camera (WFC) 
on the 2.5-meter Isaac Newton Telescope (INT).
This data release presents the first 
uniformly-calibrated source catalogue
to have been extracted from the survey,
providing single-epoch photometry
for 219 million unique sources
across 92\% of the survey.
The observations were carried out between 2003 and 2012
at a median seeing of 1.1 arcsec
to a depth of $r=21.2\pm 0.5$, $i=20.0\pm 0.3$ and \ha$=20.3\pm 0.3$
($5\sigma$ limits, Vega system).
We explain the data reduction 
and quality control procedures,
describe and test the new uniform photometric calibration,
and detail the construction of the source catalogue
and its quality warning flags.
We find that the new calibration is accurate to
$\sigma=0.03$ mag.
Finally, we demonstrate the ability of the 
catalogue's unique
$r$-$i$/$r$-\ha\ colour-colour diagram to
(i) characterise stellar populations and extinction regimes
towards different Galactic sightlines
and (ii) select reliable candidate \ha-emission line objects.
The catalogue which accompanies this paper
provides the much-needed visible-light counterpart
to several infrared surveys of the Galactic Plane,
including many poorly-studied star-forming regions and ``bubbles'',
and provides images for some of the most
crowded regions to be faced by Gaia.
\end{abstract}

\begin{keywords}
catalogues, surveys, stars: emission line, Be, Galaxy: stellar content
\end{keywords}

\section{Introduction}

Since the first data release in 2008, 
the INT/WFC Photometric H$\alpha$ Survey 
of the Northern Galactic Plane (IPHAS)
has provided new insights into the contents and structure 
of our own backyard, the Milky Way. 
The original motivation for undertaking 
this large-scale programme of observation
-- spanning almost a decade 
and using more than 300 nights 
at the Isaac Newton Telescope (INT) in La Palma -- 
was to provide the digital update 
to the photographic northern H$\alpha$ surveys 
of the mid-20th century. 
By increasing the sensitivity 
with respect to these previous surveys 
by a factor $\sim$1000 (7 magnitudes), 
it was envisaged that IPHAS would allow 
the limited bright samples of Galactic emission line objects 
available at the outset \citep[e.g.][]{Kohoutek1999}, 
to be extended into larger, deeper, more statistically-robust samples 
that in turn could better inform our understanding 
of the early and late stages of stellar evolution. 
This aim has begun to be realised through a
range of published studies including: 
a preliminary catalogue of candidate emission line objects \citep{Witham2008};
discoveries of new northern symbiotic stars \citep{Corradi2008, Corradi2010}; 
new cataclysmic variables \citep{Witham2007}; 
new groups of young stellar objects \citep{Vink2008,Barentsen2011a};
along with discoveries of new and remarkable planetary nebulae 
\citep{Mampaso2006, Corradi2011, Viironen2011}.

Over the years it has become apparent that the legacy of IPHAS 
will extend beyond the traditional \ha\ applications 
of identifying emission line stars and nebulae. 
Through the provision of $r$, $i$ broadband photometry 
alongside H$\alpha$ data,
IPHAS has created the opportunity 
to study Galactic Plane populations 
in a new way.
For example, the survey’s unique $r-H\alpha$ colour, 
when combined with $r-i$,
has been shown to provide simultaneous constraints 
on intrinsic stellar colour and interstellar extinction \citep{Drew2008}. 
This has opened the door 
to a wide range of Galactic science applications, 
including the mapping of extinction across the Plane in three dimensions
and the probabilistic inference of stellar properties
\citep{Sale2009, Sale2010, Giammanco2011, Sale2012, Barentsen2013}. 
In effect, the availability of narrowband H$\alpha$ 
alongside $r, i$ magnitudes
provides coarse spectral information for huge samples of stars 
which are otherwise too faint or numerous 
to be targeted by spectroscopic surveys.
For such science applications to succeed however, 
it is vital that the imaging data is transformed 
into a homogeneously calibrated photometric catalogue, 
in which quality problems 
and duplicate detections of the same source 
are flagged. 

The first release of IPHAS data, 
covering roughly half the survey footprint,
was made in late 2007 \citep{Gonzalez-Solares2008}. 
At the time the data were insufficiently complete 
to support a homogeneously calibrated source catalogue.
The goal of this paper is to present the next release 
that takes the coverage up to over 90 percent of the survey area 
and includes a uniform calibration.
In this work we aim to
(i) explain the data reduction 
and quality control procedures that were applied,
(ii) describe and test the new global photometric calibration, and 
(iii) detail the construction and demonstrate the use 
of the source catalogue that has been extracted 
from the newly re-calibrated data.

In \S\ref{sec:observations} we start by recapitulating the key points
of the survey design and strategy.
In \S\ref{sec:reduction} we describe the data reduction
and quality control procedures.
In \S\ref{sec:calibration} we explain the uniform re-calibration
and test our results against the Sloan Digital Sky Survey (SDSS).
In \S\ref{sec:catalogue} we explain how the source catalogue was compiled.
In \S\ref{sec:discussion} we discuss the properties of the catalogue,
and finally in \S\ref{sec:demonstration} we demonstrate
the health of the release by demonstrating its scientific exploitation.
In \S\ref{sec:conclusions} we conclude and outline
our future ambitions.


\section{Observations}
\label{sec:observations}

The detailed properties of the IPHAS observing programme 
have been presented before 
by \citet{Drew2005} and \citet{Gonzalez-Solares2008}. 
To set the stage for this release, we briefly remind of some key points.
IPHAS is a 1800~sq. deg. imaging survey of the northern Galactic Plane, 
providing images and photometry in Sloan $r, i$ 
along with narrowband H$\alpha$. 
It is carried out using the Wide Field Camera (WFC) 
on the 2.5-meter Isaac Newton Telescope (INT) in La Palma. 
It is the first digital survey to offer comprehensive CCD photometry
of point sources in the Galactic Plane at visible wavelengths, 
and does so down to a limiting magnitude of $\sim$20th.
The IPHAS footprint on the northern sky spans a box 
of roughly 180 by 10 degrees, 
taking in the entire northern Galactic Plane 
at latitudes $-5^{\circ} < b < +5^{\circ}$ 
and longitudes $30^{\circ} < l < 215^{\circ}$.

The Wide Field Camera is a mosaic of 4 CCDs 
that captures a sky area of close to 0.29 square degrees.
To cover the entire northern Plane with some overlap,
the survey area was divided into 7635 telescope pointings.
Each of these pointings is accompanied by an offset position
at a displacement of $+$5 arcmin in declination 
and $+$5 arcmin in right ascension,
to deal with inter-CCD gaps, detector imperfections,
and to enable quality checks. 
The basic unit of observation hence
amounts to $2 \times 3$ exposures, 
in which each of the 3 survey filters is exposed at 2 offset sky positions, 
within an elapsed time of 10 minutes.
We shall refer to the unit of 3 exposures at the same position 
as a \emph{field},
and the combination of two fields at a small offset as a \emph{field pair}.
The survey hence contains 15270 fields grouped into 7635 field pairs.
To achieve the desired survey depth of 20th magnitude or fainter, 
the filter exposure times were set at 120 sec (narrowband \ha), 
30 sec ($r$) and 10 sec ($i$)
in the majority of the survey observations.\footnote{In 2003 
the $r$-band exposure time was 10~sec instead of 30~sec,
and since Oct 2010 the $i$-band exposure time 
has been increased from 10~sec to 20~sec.}

Data-taking began in the second half of 2003, 
and every field had been observed at least once by the end of 2008.
At that time only 76 percent of the field pairs 
satisfied our minimum quality criteria however,  
often due to the effects of clouds, poor seeing, or technical faults
(the quality criteria will be detailed in the next section). 
Since then, a programme of repeat observations has been in place 
to improve data quality. 
As a result, 92\% of the survey 
now benefits from quality-approved data.
The most recent observations which are part of this release
were obtained in November 2012.

\begin{figure*}
        \includegraphics[width=1\linewidth]{./plots/footprint/footprint_small.png}
        \caption{Survey area showing the footprints
        of all the quality-approved IPHAS fields
        which have been included in this data release.
        The area covered by each field has been coloured black
        with a semi-transparent opacity of 20\%,
        such that regions where fields overlap are darker.
        The IPHAS strategy is to observe each field twice
        with a small offset,
        and hence the vast majority of the area 
        is covered twice (dominant gray colour).
        There are small overlaps between all the neighbouring fields
        which can be seen as a honeycomb-style pattern
        of dark gray lines across the survey area.
        Regions with incomplete data are apparent as white gaps (no data) 
        or as the lightest shade of gray
        (denoting that only the offset position is missing).}
        \label{fig:footprint}
\end{figure*}

Figure~\ref{fig:footprint} shows the footprint 
of the quality-approved observations included in this work. 
The fields which remain missing 
-- covering 7 percent of the survey area --
are predominantly located towards the Galactic anti-center 
at $l > 120^o$.
Fields at these longitudes can only be accessed from La Palma 
in the months of November-December,
which is when the weather and seeing conditions are often poor
at the INT and observing attempts have failed repeatedly.
To enable the survey to be brought to completion, 
a decision was made recently to limit repeats in this area 
to individual fields requiring replacement,
i.e. fresh observations in all 3 filters may only be obtained 
at one of the two offset positions, 
if the data for the partner offset has already passed quality control.  
The catalogue is structured such that it is clear 
where a contemporaneous observation of both halves of a field pair
is not available.


\section{Data reduction and quality control}
\label{sec:reduction}

\subsection{Initial pipeline processing}

All raw data obtained with the INT were transferred
to the Cambridge Astronomical Survey Unit (CASU) 
for initial processing and archival.
The procedures used by CASU were originally devised
for the INT Wide Field imaging Survey \citep[WFS;][]{McMahon2001,Irwin2005},
which was a 200 deg$^2$ survey programme carried out 
between 1998 and 2003 after the WFC was commissioned.
Because IPHAS uses the same telescope and camera combination,
we have been able to benefit from the existing WFS pipeline.
A detailed description of the processing steps 
is found in \citet{Irwin2001}.
Its application to IPHAS has previously been described
by \citet{Drew2005} and \citet{Gonzalez-Solares2008}
and much of the source code is available 
on line\footnote{http://casu.ast.cam.ac.uk/surveys-projects/software-release}. 
In brief, the pipeline takes care of bias subtraction,
linearity correction, flat-fielding,
gain correction and de-fringing.

The reduced images are then stored in a multi-extension FITS file 
with a primary header describing the characteristics
(position, filter, exposure time, etc.) 
and four compressed image extensions 
corresponding to each of the four CCDs.
Source detection and characterisation is then carried out 
using the \textsc{imcore} tool \citep{Irwin1985,Irwin1997}.
The flux of each source is measured using both
the peak pixel height (i.e. a square 0.33$\times$0.33\arcsec\ aperture)
as well as a series of circular apertures of increasing diameter 
(1.2\arcsec, 2.3\arcsec, 3.3\arcsec, 4.6\arcsec\ and 6.6\arcsec).

The local background levels are estimated 
by computing the sigma-clipped median
flux in a grid of 64$\times$64 pixels (21$\times$21\arcsec)
across the image,
which is then interpolated to obtain an estimate 
of the background level at each pixel.
These sky levels are subtracted from the aperture photometry and
-- when required --
a deblending routine is applied which also attempts to remove
the contamination from any other nearby sources.
Whilst this approach works very well 
across the vast majority of the survey area,
the Galactic Plane unavoidably contains crowded regions 
with large numbers of overlapping sources
or rapidly spatially-varying nebulosity,
in which case aperture photometry must always be interpreted 
with caution.
In \S\ref{sec:catalogue} we will explain that overlapping
sources to which the deblending routine was applied 
are flagged in the catalogue using the \emph{deblend} warning flag.

Finally, an astrometric solution is determined
based on the 2MASS point source catalog \citep{Skrutskie2006},
which itself is calibrated 
in the International Celestial Reference System (ICRS).
A provisional photometric calibration is also provided 
based on the average zeropoint
determined from a set of standard stars observed in the same night.
Sources are classified morphologically
-- stellar, galaxy or noise --
based on the curve-of-growth determined
from measuring the source intensity in a series of growing apertures.
Finally, the resulting source detection tables are also stored 
in multi-extension FITS files.

At the time of preparing DR2,
the CASU pipeline had processed
74\,195 IPHAS exposures 
in which a total of 1.9~billion \emph{candidate sources} were detected 
at the sensitive default detection level of 1.25\,$\sigma$
-- unavoidably including spurious detections, artefacts and
duplicate detections 
(in \S\ref{sec:catalogue} we will explain
how these have been removed or flagged).
The pipelined data set -- comprising 2.5~terabyte of FITS files --
was then transferred to the University of Hertfordshire
for the purpose of transforming the raw
detection tables into a reliable source catalogue which is 
(i) quality-controlled,
(ii) homogeneously calibrated, and 
(iii) contains user-friendly columns and warning flags.
It is these post-processing steps which are explained below.


\subsection{Quality control}
\label{sec:qc}

Observing time for IPHAS was obtained
on a semester-by-semester basis
through the traditional time allocation committees 
of the Isaac Newton Group of telescopes,
which are competitive and invariably over-subscribed.
For this reason, we attempted to utilise 
\emph{all} the nights allocated to IPHAS,
even those which were partially or entirely non-photometric
or otherwise affected by technical problems 
(e.g. electronic noise or telescope tracking problems).
Any unsuitable data that was taken as a result
of this strategy was subsequently flagged and rejected
using a series of seven quality criteria,
which ensure a reliable and homogeneous level of quality
across the data release:

\begin{figure}
    \begin{minipage}[b]{\linewidth}
        \includegraphics[width=\textwidth]{./plots/depth_r.pdf} 
    \end{minipage}
    \begin{minipage}[b]{\linewidth}
        \includegraphics[width=\textwidth]{./plots/depth_i.pdf} 
    \end{minipage}
    \begin{minipage}[b]{\linewidth}
        \includegraphics[width=\textwidth]{./plots/depth_h.pdf} 
    \end{minipage}
    \caption{Distribution of the 5$\sigma$ limiting magnitude
             across all quality-approved survey fields
             for $r$ (top), $i$ (middle) and \ha\ (bottom).
             Fields with a limiting magnitude brighter than
             20th ($r$) or 19th (\ha/$i$) were rejected
             from the data release.
             The $r$-band depth is most sensitive 
             to the presence of the moon above the horizon, 
             which is evidenced by the wide and bi-model shape
             of its distribution.}
    \label{fig:depth}
\end{figure}

(1) \emph{Depth.} 
We discarded any exposures for which the $5\sigma$ limiting magnitude 
was worse than 20th magnitude in the $r$-band
or worse than 19th in $i$ or \ha. 
Such data were typically obtained during poor weather or full moon.
Most observations fared significantly better than these limits.
Figure~\ref{fig:depth} presents the distribution of limiting magnitudes
for all quality-approved fields,
which shows a mean depth of 
$21.2\pm0.5$ ($r$), $20.0\pm0.3$ ($i$) and $20.3\pm0.3$ (\ha).
We found that the depth achieved depended 
most strongly on the presence of the moon,
which was above the horizon during 62 percent 
of our observations and explains the wide and bi-modal shape
of the $r$-band limiting magnitude distribution 
(top panel in Fig.~\ref{fig:depth}).
In contrast, the depth attained in $i$ and \ha\ 
is less sensitive to moonlight
and the distribution of their depths
is hence more narrow
(middle and bottom panel in Fig.~\ref{fig:depth}).

Our minimum limiting magnitude criteria
have led us to exclude 9\% of the pipelined data.
We note that much of the excluded data may nevertheless be useful
for e.g. time-domain studies of bright stars.
The detection tables for any such discarded data are made
available through our website (www.iphas.org)
but are ignored in what follows.

(2) \emph{Ellipticity.} 
The ellipticity of a point source,
defined as $e = 1 - b / a$ 
with $b$ the semi-minor and $a$ the semi-major axis,
is a morphological measure of the elongation of the point spread function.
It is expected to be zero (circular) across the field 
in a perfect imaging system,
but it is slightly non-zero in any real telescope data 
due to optical distortions and tracking errors.
The mean ellipticity across a field in the IPHAS data set 
is $0.09\pm0.04$.
There have been sporadic episodes with higher ellipticities however
due to mechanical glitches in the telescope tracking system.
For this reason, we rejected exposures in which the mean ellipticity
across the detector exceeded $e > 0.3$,
which is when the photometric measurements delivered by the pipeline
were found to become degraded.
Only 0.4\% of the exposures were discarded on this basis.

(3) \emph{Seeing.} 
The survey originally aimed to obtain data 
at seeing better than 1.7 arcsec.
This target is currently attained across 86\% of the footprint,
in particular at early longitudes,
e.g. 92\% of the fields at $l<120^\circ$ are better than 1.7 arcsec.
Figure~\ref{fig:seeing} presents the distribution
of the mean seeing for all the quality-approved fields.
We find a median value of 1.1~arcsec in $r$/\ha\
and 1.0~arcsec in $i$.
In the $r$-band, 90\% of the data is better than 1.5~arcsec
and 10\% is better than 0.8~arcsec.
To improve the area covered by this data release,
we have decided to include the small fraction of data
that was obtained under seeing up to 2.5 arcsec,
so that only 1 per cent of the pipelined exposures
had to be excluded.
In \S\ref{sec:catalogue} we will explain
that the catalogue is structured
such that the information on the seeing is included,
and that the photometry listed
is based on the exposures with the best-available seeing.

\begin{figure}
    \begin{minipage}[b]{\linewidth}
        \includegraphics[width=\textwidth]{./plots/seeing_r.pdf} 
    \end{minipage}
    \begin{minipage}[b]{\linewidth}
        \includegraphics[width=\textwidth]{./plots/seeing_i.pdf} 
    \end{minipage}
    \begin{minipage}[b]{\linewidth}
        \includegraphics[width=\textwidth]{./plots/seeing_ha.pdf} 
    \end{minipage}
    \caption{Seeing distribution across all
             quality-approved survey fields
             for $r$ (top), $i$ (middle) and \ha\ (bottom).}
    \label{fig:seeing}
\end{figure}


(4) \emph{Photometric repeatability.} 
The IPHAS field-pair observing strategy 
ensures that every pointing is immediately followed 
by an offset pointing at a displacement of $+$5 arcmin in Dec 
and $+$5 arcmin in RA.
This allows pairs of images to be checked 
for the presence of clouds or electronic noise.
To exploit this information,
the overlap regions of all field pairs were systematically cross-matched
to verify the consistency of the photometric measurements
for stars observed in both pointings.
We rejected field pairs in which more than 2\% of the stars 
showed an inconsistent measurement at the level of 0.2 mag,
or more than 26\% were inconsistent at the level of 0.1 mag.
These limits were determined empirically by inspecting
the images and photometry by eye.
11\% of the data was rejected as part of this step.

(5) \emph{Source density mapping.}
Spatial maps showing the number density of the detected sources
down to 20th magnitude were created to verify the health
of the data and to check for unexpected artefacts.
%(e.g. bright satellite trails)
In particular, we created density maps
which showed the number of \emph{unique} sources
obtained by cross-matching the detection tables of
all three bands with a maximum matching distance of 1 arcsec.
The success of such a cross-matching procedure crucially depends
on the accuracy of the astrometric solution in each band,
and hence we were able to detect and correct
exposures with poor astrometry by inspecting the density map
for spurious density variations.

(6) \emph{Visual examination.}
All images and their associated photometric colour/magnitude diagrams
were inspected by a team of 20 survey team members, 
such that each image in the data release 
was looked at by at least three different pairs of eyes.
Images deemed unsuitable were flagged, investigated and
excluded from the release if necessary. 
6\% of the attempts to observe a field
were placed on a \emph{black-list}
for this reason, most commonly due to the obvious presence
of clouds or extreme levels of scattered moonlight.
Such fields were often rejected as part of 
one or more of the above quality criteria as well.

(7) \emph{Contemporaneous field data.} 
Finally, only exposures which are part of a sequence 
of three consecutive images of the same field (H$\alpha$/$r$/$i$) 
were considered for inclusion in the release. 
This ensures that the three bands for a given field
are observed at nearly the same time --  
essentially always within 5 minutes.
An exception was made for 9 fields where the three exposures 
could not be obtained within the same night
but for which the time gap did not exceed 48 hours.
We note that the exact epoch of the magnitude in each band
is included in the source catalogue
(columns \emph{rMJD}, \emph{iMJD}, \emph{haMJD}).

The above criteria were satisfied 
for 14115 out of the 15270 fields (92\%).
In a few cases more than one successful attempt to observe
a field was available due to the fact that slightly stricter
quality criteria were adopted in the initial years of the survey.
In such cases, only the attempt 
with the best seeing and depth was selected
for inclusion in the release, because the focus 
of this work is to deliver the most reliable
measurement at a single epoch.
Those interested in any of the rejected data 
may nevertheless access the full set of detection tables 
on line.


\section{Photometric calibration}
\label{sec:calibration}

Having obtained a quality-approved set of observations,
we now turn to the problem of placing the data
onto a uniform photometric scale.

\subsection{Provisional nightly calibration}

For the purpose of providing an initial calibration 
of the $r$ and $i$ broadband fluxes,
photometric standard fields were observed every night.
The standards were chosen from a list based on 
the \cite{Landolt1992} and Stetson (http://cadcwww.dao.nrc.ca/standards) 
objects.
Two or three standard fields were observed 
during the evening and morning twilight,
and at intervals of 2-3 hours throughout the night.
The CASU pipeline automatically identified the observed standards 
and used them to determine a sigma-clipped average zeropoint \textsc{magzpt}
for each night and filter,
such that the number counts $DN$ 
in the pipeline-corrected CCD frames
relates to a magnitude $m$ as:
\begin{equation}
\begin{split}
   m  = & \textsc{magzpt} - 2.5 \log_{10}( DN / \textsc{exptime} ) \\
 &  - \textsc{extinct}\cdot(\textsc{airmass}-1) - \textsc{apcor} - \textsc{percorr}
\label{eqn:mag}
\end{split}
\end{equation}
where \textsc{exptime} is the exposure time in seconds,
\textsc{extinct} is the atmospheric extinction coefficient 
(typically 0.09 for $r$ and 0.05 for $i$ in La Palma),
\textsc{airmass} is the normalised optical path length 
through the atmosphere and
\textsc{apcor} is a correction for the flux
lost outside of the aperture used.
Finally, \textsc{percorr} is a correction based on the difference
between the median dark sky for a CCD against the median for all the CCDs 
and as such is an ancillary correction 
to account for sporadic gain variations. 
All these quantities correspond to header keywords in the 
multi-extension FITS files produced by the CASU pipeline.

The zeropoint was determined such that the resulting magnitude system
is in the WFC system that uses the SED of Vega 
as the zero colour reference object. 
Colour equations were used to transform between the IPHAS passbands 
and the Johnson-Cousins system 
of the published standard star photometry.
The entire procedure has been found to deliver zeropoints which 
are accurate at the level of 1-2\% 
in photometric conditions \citep{Gonzalez-Solares2011}.

Unlike the broadbands, 
standard star photometry is not available in the literature 
for the H$\alpha$ passband
and hence there is no formally recognised flux scale 
for the narrowband.
We can specify here however 
that the integrated in-band energy flux for Vega 
in the IPHAS \ha\ filter 
is $1.52 \times 10^{-7}$ ergs\,cm$^{-2}$\,s$^{-1}$ 
at the top of the Earth's atmosphere,
which is the flux obtained by folding 
Vega's SED with the filter throughput curve 
corrected for atmosphere and detector quantum efficiency
\citep[using the method explained by][]{Drew2005}.
This is 3.08 magnitudes less than the flux captured 
by the much broader $r$ band
which includes the \ha\ band.
Hence to assure zero colour relative to the broadbands,
we set the default zeropoint for the narrowband to be:
\begin{equation}
\textsc{magzpt}_{H\alpha} = \textsc{magzpt}_r - 3.08.
\label{eqn:zpha}
\end{equation}

\subsection{Uniform re-calibration}

Despite the best efforts made to obtain a nightly calibration,
large surveys naturally possess field-to-field variations
at the level of 0.1 mag
due to atmospheric fluctuations during the night
and imperfections in the pipeline or the instrument
(e.g. the WFC is known to suffer from sporadic glitches
in the timing of exposures).
Such variations need to be corrected for 
during a global re-calibration procedure.
Notable past examples include the global re-calibration 
of the Two Micron All Sky Survey \citep[2MASS;][]{Nikolaev2000},
the Sloan Digital Sky Survey \citep[SDSS;][]{Padmanabhan2008}
and the Panoramic Survey Telescope 
and Rapid Response System survey \citep[Pan-STARRS;][]{Schlafly2012},
which all achieved photometry 
that is globally consistent to within 1--2\%.

Surveys which observe identical stars at different epochs
can use the repeat measurements to ensure a uniform calibration.
For example, 2MASS attained its global calibration
by observing six standard fields each hour, 
allowing zeropoint variations to be tracked 
over very short timescales \citep{Nikolaev2000}.
Alternatively, the SDSS and PanSTARRS surveys could benefit
from revisiting regions in their footprint to 
carry out a so-called \emph{ubercalibration} procedure,
in which multiple measurements of stars at different epochs
are used to fit the calibration parameters
\citep{Padmanabhan2008,Schlafly2012}.

Unfortunately these schemes cannot be applied
directly to IPHAS
for two reasons. 
Firstly, the survey was carried out 
in competitive observing time
on a non-dedicated telescope, 
rendering the 2MASS approach 
of observing standards at a very high frequency
prohibitively expensive
-- in part because standard fields 
are very scarce within the Galactic Plane.
Secondly, the aim of IPHAS is to obtain magnitudes at a single epoch
and hence stars are not normally observed at more than one epoch,
unless they happen to fall on the narrow overlap region 
between two neighbouring fields.

Although the IPHAS data does contain a significant number of 
inter-field repeat measurements
due to the small overlaps
between neighbouring fields,
we have found the information contained
in these regions to be insufficient
to constrain the calibration parameters.
This is because photometry at the extreme edges of the WFC
-- where neighbouring fields overlap -- 
is itself prone to systematics at the level of 1-2\%.
The cause of these errors is thought to include 
the use of twilight sky flats in the pipeline,
which are known to be imperfect for calibrating stellar photometry 
due to stray light and vignetting \citep[e.g.][]{Manfroid1995}.
Moreover, the illumination correction in the overlap regions
is affected by a radial geometric distortion,
which causes the pixel scale towards the edges 
to increase \citep{Gonzalez-Solares2011}.
Although these systematics are reasonably small within a single field,
they can combine to cause artificial zeropoint gradients 
across the survey
when they are used to constrain a global calibration
without external constraints.

For these reasons, we have decided not to depend
on an ubercalibration-type scheme alone,
and have instead opted to exploit an external reference survey
-- where available --
to bring the majority of our data onto a uniform calibration.
This is explained next.

\subsubsection{Correcting zeropoints using APASS}

We have been able to benefit from the
AAVSO Photometric All-Sky Survey
(APASS; http://www.aavso.org/apass)
to bring the vast majority of the survey 
onto a uniform scale.
Since 2009,
APASS has been using two 20~cm-astrographs
to survey the entire sky down to $\sim$17th magnitude
in five filters which include Sloan $r$ and $i$.
The most recent catalogue available 
at the time of preparing this work was APASS DR7,
which provided data across roughly half of the IPHAS footprint (Fig.~\ref{fig:apass_r}).
The photometric accuracy of APASS is currently estimated 
to be at the level of 3\% (Henden, private communication),
which is significantly better 
than the provisional calibration of IPHAS
for which we estimate the $1\sigma$-error to be $\sim$10\%.
APASS achieves its uniform accuracy 
by measuring each star at least two times in photometric conditions
-- along with ample standard fields --
using the large $3\times3$ square degrees field of view of its detectors.

With the aim of bringing IPHAS to a similar accuracy of $\sim$3\%,
we used the APASS catalogue to identify and adjust all IPHAS fields 
which showed a magnitude offset larger than 3\% against APASS.
For this purpose,
the $r$- and $i$-band detection tables of each IPHAS field
were cross-matched against the APASS DR7 catalogue 
using a maximum matching distance of 1~arcsec.
The magnitude range was limited to
$13<r_{\rm APASS}<16.5$ and $12.5<i_{\rm APASS}<16.0$
in order to avoid sources 
brighter than the IPHAS saturation limit on one hand, 
and to avoid sources near the faint detection limit of APASS 
on the other hand.

The resulting set of $\sim$220\,000 cross-matched stars were then used 
to derive APASS-to-IPHAS magnitude transformations
using a linear least-squares fitting routine, 
which iteratively removed $3\sigma$-outliers to improve the fit.
The solution converged to:
\begin{align} 
r_{\rm IPHAS} = r_{\rm APASS} - 0.121 + 0.032(r-i)_{\rm APASS} \label{eqn:apass_r} \\
i_{\rm IPHAS} = i_{\rm APASS} - 0.364 + 0.006(r-i)_{\rm APASS} \label{eqn:apass_i}
\end{align}
The root mean square (rms) residuals of these transformations 
are 0.041 and 0.051, respectively.
The small colour terms in the equations
indicate that the $r$ and $i$ filters 
are very similar in both surveys.
The transformations include a large fixed offset
which is simply due to the fact that 
APASS magnitudes are given in the AB system
and IPHAS uses magnitudes relative to Vega.
Separate transformations were derived for sightlines 
with varying extinction properties to investigate the robustness
of the transformations with respect to different reddening regimes,
but the variations at these different
sightlines were found to be insignificant.
This is not surprising because heavily reddened objects 
are naturally scarce at $r<16$.

Having transformed APASS magnitudes into the IPHAS system,
we then computed the median magnitude offset 
for each field which contained at least 30 cross-matched stars.
This was the case for 48\% of the fields.
The mean offset was found to be
$0.014\pm0.104$ for $r$ and $0.007\pm0.108$ for $i$
(Table~\ref{tbl:offsets_before}).
A total of 4596 fields showed a median offset
exceeding $\pm$0.03~mag in either $r$ or $i$.

We then applied the most important step in our re-calibration scheme,
which is to adjust the provisional zeropoints of the 4596 aberrant fields
such that their offset is brought to zero.
This allowed the mean IPHAS-to-APASS offset 
to be brought down to $0.000\pm0.011$ in both $r$ and $i$
(Table~\ref{tbl:offsets_after}).
The procedure of fitting magnitude transformations and
correcting the zeropoints was repeated a few times to ensure 
convergence, which was essentially reached after the first iteration.

\begin{figure*}
    \includegraphics[width=\textwidth]{plots/calibration/APASS-IPHAS-DR2_rshift.pdf} 
    \includegraphics[width=\textwidth]{plots/calibration/colourbar_apass_r.pdf} 
    \caption{Median magnitude offset in the $r$ band between IPHAS and APASS,
             plotted on a field-by-field basis
             prior to the re-calibration procedure.
             Each square represents an IPHAS field
             which contains at least 30 stars with a counterpart
             in the APASS DR7 catalogue.
             The colours denote the median
             IPHAS-APASS magnitude offset in each field,
             which was computed after applying the APASS-to-IPHAS
             transformation to the APASS magnitudes (Eqn.~\ref{eqn:apass_r}).}
        \label{fig:apass_r}
    \vspace{1cm}
    \includegraphics[width=\textwidth]{plots/calibration/SDSS-IPHAS_rshift.pdf}
    \includegraphics[width=\textwidth]{plots/calibration/colourbar_sdss_r.pdf} 
    \caption{Median magnitude offset in the $r$ band
             between IPHAS and SDSS after the re-calibration
             procedure was applied.
             Again, each square represents an IPHAS field
             which contains at least 30 stars
             with a counterpart in the SDSS catalogue.
             The colours denote the median IPHAS-SDSS magnitude offset
             in each field,
             which was computed after applying the SDSS-to-IPHAS
             transformation to the SDSS magnitudes (Eqn.~\ref{eqn:sdss_r}).}
    \label{fig:sdss_r}
\end{figure*}

\begin{figure*}
	\vspace{2cm}
    \includegraphics[width=\textwidth]{plots/calibration/APASS-IPHAS-DR2_ishift.pdf} 
    \includegraphics[width=\textwidth]{plots/calibration/colourbar_apass_i.pdf} 
    \caption{Same as Figure~\ref{fig:apass_r} for the $i$-band.}
        \label{fig:apass_i}
    \vspace{1cm}
    \includegraphics[width=\textwidth]{plots/calibration/SDSS-IPHAS_ishift.pdf}
    \includegraphics[width=\textwidth]{plots/calibration/colourbar_sdss_i.pdf} 
    \caption{Same as Figure~\ref{fig:sdss_i} for the $i$-band.}
    \label{fig:sdss_i}
\end{figure*}

\begin{table}
        \begin{center}
                \begin{tabular}{lcc}
                        \toprule
                         {\bf Before re-calibration} & Mean & $\sigma$  \\
                        \midrule
                        $r$ (IPHAS - APASS) & +0.014 & 0.104 \\
                        $i$ (IPHAS - APASS) & +0.007 & 0.108 \\
                        $r$ (IPHAS - SDSS) & +0.016 & 0.088 \\
                        $i$ (IPHAS - SDSS) & +0.010 & 0.089 \\
                        \bottomrule
                \end{tabular}
        \caption{Mean magnitude offsets for objects
                 crossmatched between IPHAS and APASS/SDSS
                 before the uniform re-calibration.
                 Eqns.~\ref{eqn:apass_r}-\ref{eqn:sdss_i} were applied
                 to the APASS/SDSS magnitudes to bring them into the
                 Vega-based IPHAS system prior to computing
                 the offsets.)
                 }
        \label{tbl:offsets_before}

                \begin{tabular}{lcc}
                        \toprule
                         {\bf After re-calibration} & Mean & $\sigma$ \\
                        \midrule
                        $r$ (IPHAS - APASS) & +0.000 & 0.011\\
                        $i$ (IPHAS - APASS) & +0.000 & 0.011 \\
                        $r$ (IPHAS - SDSS)  & -0.001 & 0.029\\
                        $i$ (IPHAS - SDSS) & -0.002 & 0.032 \\
                        \bottomrule
                \end{tabular}
        \caption{Same as Table~\ref{tbl:offsets_before}
                 but computed after the uniform re-calibration
                 was carried out.}
        \label{tbl:offsets_after}
        \end{center}
\end{table}

\subsubsection{Adjusting fields not covered by APASS}

At the time of writing, the APASS catalogue did not provide 
sufficient coverage for 7359 of the fields in our data release.
Fortunately, these fields are pre-dominantly located 
in the early part of the Galactic Plane (Fig.~\ref{fig:apass_r}),
which were typically observed during the summer months
when photometric conditions are more prevalent at the telescope.
These remaining fields have nevertheless 
been brought onto the same uniform scale 
by employing an ubercalibration-style scheme, explained below,
which minimises the magnitude offsets between stars
located in the overlap regions with neighbouring fields.
Although we explained earlier that these overlap regions
are prone to systematics, the use of APASS enabled us to
keep the zeropoints of roughly half of the fields fixed,
which avoids these systematics from combining to introduce
artificial gradients across the survey.
In the next section we will show this to be true by validating
our calibration against SDSS.

A general solution to the problem of minimising the magnitude
differences between overlapping frames has previously been
described by~\citet{Glazebrook1994}.
In brief, there are two fundamental quantities to be
minimised between each pair of overlapping exposures, 
denoted by the indices $i$ and $j$. 
Firstly, the mean magnitude difference between stars in the overlap
region $\Delta_{ij}=\langle m_i-m_j\rangle=-\Delta_{ji}$ is a local
constraint. 
Secondly, to ensure the solution does not stray far 
from the existing calibration, 
the difference in zeropoints 
$\Delta\mathrm{ZP}_{ij}=-\Delta\mathrm{ZP}_{ji}$ 
between each pair of exposures must also be minimised.

Minimisation of these two quantities is a linear least squares problem 
because the magnitude $m$ depends linearly on the ZP (Eqn.~\ref{eqn:mag}).
Hence we can find the ZP shift to be applied to each field 
by minimising the sum:
\begin{equation}
   S = \sum_{i=1}^N \sum_{j=1}^N w_{ij} \theta_{ij} (\Delta_{ij} + a_i - a_j)^2
   \label{eqn:chi2}
\end{equation}
where $i$ denotes the exposure of interest, 
$j$ an overlapping exposure, 
$N$ the number of exposures,
$a_i$ the ZP to solve for 
and $a_j$ the ZP of an overlapping field ($\Delta\mathrm{ZP}_{ij}=a_i-a_j$), 
$w_{ij}$ are weights set to the uncertainty in $\Delta_{ij}$
and $\theta_{ij}$ is an overlap function 
equal to either 1 if exposures $i$ and $j$ overlap or 0 otherwise. 
Solving for $a_i$ is equivalent to solving $\partial
S/\partial a_i=0$ which gives the matrix equation:
\begin{equation}
   \sum_{j=1}^N A_{ij} a_j = b_j
   \label{eqn:matrix}
\end{equation}
where 
\begin{eqnarray}
   A_{ij} &=& \delta_{ij} \sum_{k=1}^N w_{jk}\theta_{jk} - w_{ij} \theta_{ij},\label{eqn:aij}\\
   b_i &=& \sum_{j=1}^N w_{ij} \theta_{ij}\Delta_{ji} = -\sum_{j=1}^N w_{ij} \theta_{ij}\Delta_{ij}.\label{eqn:bi}
\end{eqnarray}
This prescription is essentially identical to \citet{Glazebrook1994}.
%,although we changed the formulation to ensure that $A_{ij}$ is positive definite.

As explained above, we enforce a strong external constraint
on the solution by keeping the zeropoint fixed 
for the 6756 fields which have been compared
and calibrated against APASS earlier.
We will hereafter refer to these fields as \emph{anchors}.
It is asserted that the zeropoints $a_i$ of these anchor fields 
are known and not solved for,
though they do appear in the vector $b_j$ as constraints.
In addition to the APASS-based anchors, 
we selected 3273 additional anchor fields by hand
to provide additional constraints in regions not covered by APASS.
These extra anchors were deemed to have reliable zeropoints 
based on 
(i) the information contained in the observing logs,
(ii) the stability of the standard star zeropoints during the night, and
(iii) photometricity statistics provided by the Carlsberg Meridian Telescope,
which is located at $\sim$500~meter from the INT.

We then solved Eqn.~\ref{eqn:matrix} for the $r$ and $i$ bands
separately using the least-squares routine 
in Python's {\sc scipy.sparse} module for sparse matrix algebra.
This provided us with corrected zeropoints for the remaining fields,
which were shifted on average by $+0.02\pm0.11$ in $r$ 
and $+0.01\pm0.12$ in $i$ compared to their provisional calibration.

We then turned to the uniform calibration of the \ha\ data.
It is not possible to re-calibrate the narrowband 
in the same way as the broadbands,
because the APASS survey does not offer \ha\ photometry.
We can reasonably assume however,
that the corrections required for $r$ and \ha\ are identical
because the \ha\ zeropoints have been derived directly from the
$r$-band zeropoints during the provisional calibration (Eqn.~\ref{eqn:zpha}).
Moreover, the IPHAS data-taking pattern ensured 
that a field's \ha\ and $r$-band exposures
were taken at essentially the same time, 
separated only by the $\sim$30-second read-out time required by the WFC.
We have hence corrected the \ha\ zeropoints 
by re-using the zeropoint adjustements that were derived for the $r$ band
in the earlier steps.
An exception was made for 3101 fields
for which our quality-control routines revealed
zeropoint variations which exceeded 0.03~mag 
between consecutive fields,
which indicates non-photometric conditions.
For good practice, the \ha\ zeropoints of these fields
were adjusted by solving Eqn.~\ref{eqn:matrix}
rather than linking them directly to the $r$-band shift.

\subsection{Testing the calibration against SDSS}

Having re-calibrated all fields to an expected accuracy of 3\%,
we then used an independent survey to validate the results.
We have been able to exploit SDSS Data Release 9 \citep{Ahn2012}
for this purpose.
SDSS DR9 provides several strips at low
Galactic latitudes,
providing data across 18\% of the fields in our data release.
We cross-matched IPHAS fields against 
stars marked as reliable in the SDSS catalogue\footnote{
We used the CasJobs facility located at http://skyserver.sdss3.org/CasJobs
to obtain photometry from the SDSS {\sc photoprimary} table 
with criteria {\sc type = star}, {\sc clean = 1} and {\sc score $>$ 0.7}.}
in the same way as we did for APASS earlier,
with the exception of using fainter magnitude ranges of 
$15<r_{\rm SDSS}<18.0$ and $14.5<i_{\rm SDSS}<17.5$.
This provided us with a set of 1.2 million cross-matched stars.

Colour transformations were obtained using a sigma-clipped linear least squares fit:
\begin{eqnarray}
r_{\rm IPHAS} = r_{\rm SDSS} - 0.093 - 0.044(r-i)_{\rm SDSS} \label{eqn:sdss_r} \\
i_{\rm IPHAS} = i_{\rm SDSS} - 0.318 - 0.095(r-i)_{\rm SDSS}. \label{eqn:sdss_i}
\end{eqnarray}
The rms residuals of these transformations are 0.045 and 0.073, respectively.
The equations are similar to the ones
previously determined for APASS,
although the colour terms are slightly larger;
the throughput curve of the SDSS $i$-band filter 
appears to be somewhat more sensitive at longer wavelengths
compared to both the IPHAS and APASS filters.

These global transformations were deemed adequate
for the purpose of validating our uniform calibration in a statistical sense.
Separate equations were derived towards different sightlines
to investigate the effects of varying reddening regimes.
The transformation coefficients were found 
to show some variation towards lowly reddened areas,
which have relatively few numbers
of (intrinsically) red objects at $r-i > 1$ 
which can skew the colour term.
The vast majority of red objects in the global sample
are those in highly reddened areas however,
which agree well with the global transformations
and dominate the statistical appraisal of our calibration.
%The small number of intrinsically red objects in lowly reddened fields
%have no impact on the statistical appraisal of our calibration.

Having transformed SDSS magnitudes into the IPHAS system,
we then computed the median magnitude offset for each IPHAS field
which contained at least 30 objects with a counterpart
in the SDSS catalogue.
This was the case for 2602 fields.
The median offsets for each of these fields
are shown in Figs.~\ref{fig:sdss_r}-\ref{fig:sdss_i}.
The mean offset and standard deviation found 
was $-0.001\pm0.029$ for $r$
and $-0.002\pm0.032$ for $i$ (Table~\ref{tbl:offsets_after}).
In comparison, offsets computed in the identical way
\emph{before} our re-calibration showed means
of $+0.016\pm0.088$ and $+0.010\pm0.089$ (Table~\ref{tbl:offsets_after}).
We conclude that our re-calibration procedure has
been successful in improving the
uniformity of the calibration by a factor three
and has achieved our aim of bringing the
accuracy down to the level of $\sigma=0.03$~mag.

We warn that the SDSS comparison reveals a number of fields with offsets
exceeding 0.05~mag (523 fields) or even 0.1~mag (18 fields).
Such sporadic outliers are consistent with the tails of a Gaussian
with mean $\sim0$ and $\sigma=0.03$.

In future work, we hope to draw upon
the PanSTARRS survey \citep{Schlafly2012}
to further improve the accuracy of our calibration.
At the time of preparing this work data from PanSTARRS
had not been made public yet.

\section{Source catalogue generation}
\label{sec:catalogue}

Having obtained a quality-checked 
and re-calibrated data set, 
we now turn to the problem
of transforming the observations 
into a user-friendly catalogue.
The aim of this catalogue is to detail
the best-available information for each unique source
in a convenient format,
including flags to warn about quality issues 
such as source blending and saturation.
Compiling the catalogue essentially required four steps:
\begin{enumerate}
\item the single-band detection tables 
produced by the CASU pipeline 
were augmented with new columns
and warning flags;
\item the detection tables were merged into multi-band field catalogues;
\item the overlap regions of the field catalogues 
were cross-matched to flag duplicate measurements 
and identify the best detection 
of each unique source; and
\item these primary detections
were compiled into the final source catalogue.
\end{enumerate}
Each of these four steps are now explained.

\subsection{User-friendly columns and warning flags}

As the first step, the detection tables 
were enhanced by creating new columns.
This is necessary because the tables 
generated by the CASU pipeline 
summarise the detections 
in their original CCD units,
e.g. source positions are given in pixel coordinates 
and photometry in number counts.
To transform these measurements into
user-friendly fields,
we have largely adopted the units and naming conventions
which are in use at the 
WFCAM Science Archive \citep[WSA;][]{Hambly2008}
and the VISTA Science Archive \citep[VSA;][]{Cross2012}.
These archives curate the near-infrared data from both
the UKIDSS Galactic Plane Survey \citep[GPS;][]{Lucas2008}
and the 
VISTA Variables in the Via Lactea survey \cite[VVV;][]{Minniti2010}.
Both these surveys provide high-resolution JHK photometry
in the Galactic Plane.
There is a significant degree of overlap
between the footprints of UKIDSS/GPS and IPHAS,
and hence by adopting a similar catalogue format
we hope to encourage scientific applications
which combine both data sets.

A detailed description of each column in our source catalogue
is given in Appendix~\ref{app:columns}.
In the remainder of this section we highlight the main features.

Firstly, we note that each source is uniquely identified by an
IAU-style designation of the form ``IPHAS2\ JHHMMSS.ss+DDMMSS.s''
(cf. column \emph{name} in Appendix~\ref{app:columns}),
where ``IPHAS2'' refers to the present
data release and the remainder of the string
denotes the J2000 coordinates in sexagesimal format.
For convenience, the coordinates
are also included in decimal degrees
(columns \emph{ra} and \emph{dec})
and in the Galactic coordinate system
(columns \emph{l} and \emph{b}).
We have also included an internal object identifier string 
of the form ``\#run-\#ccd-\#detection''
(e.g. ``64738-3-6473''),
which documents the INT exposure number (\#run),
the CCD number (\#ccd),
and the row number in the CASU detection table (\#detection)
-- see columns \emph{rDetectionID},
\emph{iDetectionID}, \emph{haDetectionID}.

Photometry is provided based on the 2.3-arcsec diameter circular aperture
by default (columns \emph{r}, \emph{i}, \emph{ha}).
The choice of this aperture size as the default 
is based on a trade-off between concerns 
about small number statistics and centroiding errors
for small apertures on one hand,
and diminishing signal-to-noise ratios and source confusion
for large apertures on the other hand.
The user is not restricted to this choice because
the catalogue also provides magnitudes
using three alternative aperture sizes:
the peak pixel height 
(columns \emph{rPeakMag}, \emph{iPeakMag}, \emph{haPeakMag}),
a circular 1.2-arcsec-diameter aperture 
(\emph{rAperMag1}, \emph{iAperMag1},
 \emph{haAperMag1}) and
a 3.3-arcsec-diameter aperture 
(\emph{rAperMag3}, \emph{iAperMag3},
 \emph{haAperMag3}).

Each of these magnitude measurements have been
corrected for the flux lost outside of their respective apertures,
using a correction term which is inferred from the
mean shape of the PSF measured locally in the CCD frame.
In the case of a point source,
the four alternative magnitudes are expected
to be consistent with each other
within the photon noise uncertainties
(which are given in columns \emph{rErr}, \emph{rPeakMagErr},
\emph{rAperMag1Err}, \emph{rAperMag3Err}, etc).
When this is not the case,
it is likely that the source is either
an extended object or that it has
been incorrectly measured as a result of
source blending or a rapidly spatially-varying nebulous background.
In \S\ref{sec:qualitycriteria} we will explain that the consistency
of the magnitude measurements in the different apertures
can be used as a criterion for selecting stellar objects
with reliable photometry from the catalogue.

The brightness of each object as a function of increasing
aperture size is also used by the CASU pipeline to provide
a discrete star/galaxy/noise classification flag
(\emph{rClass}, \emph{iClass}, \emph{haClass})
and a continuous stellarness-of-profile statistic
(\emph{rClassStat}, \emph{iClassStat}, \emph{haClassStat}).
For convenience, we have combined
these single-band morphological measures
into band-merged class probabilities and flags
(\emph{pStar}, \emph{pGalaxy}, \emph{pNoise},
\emph{mergedClass}, \emph{mergedClassStat})
using the merging scheme in use at the WSA\footnote{Explained at
http://surveys.roe.ac.uk/wsa/www/gloss\_m.html \#gpssource\_mergedclass
}.


Information on the quality of each detection is included 
in a series of additional columns.
We draw attention to three useful flags
which warn about the likely presence of systematics:
\begin{enumerate}
\item The \emph{saturated} column is used to flag sources
for which the peak pixel height exceeds 55000 counts,
which is typically the case for stars brighter than 12-13th magnitude.
Although the pipeline attempts to extrapolate the brightness of
saturated stars based on the shape of their PSF,
such extrapolation is prone to errors
and we do not recommend their use.
\item The \emph{deblend} column is used to flag sources 
which partially overlap with a nearby neighbour.
Although the pipeline applies a deblending procedure
to such objects, the procedure is currently applied separately
in each band and hence the $r$-$i$ and $r$-\ha\ colours
of such objects are prone to errors.
\item The \emph{brightNeighb} column is used to flag sources which are located
within 5 arcmin from an object brighter than $V=7$ 
according to the Bright Star Catalogue (BSC; Hoffleit et al. 1991), 
or within 10 arcmin if the neighbour is brighter than $V=4$.
Such very bright stars are known to cause systematic errors
and spurious detections as a result of stray light 
and diffraction spikes.
\end{enumerate}
In addition to the above, we also created warning flags for internal bookkeeping.
For example, we flagged detections which fell in the strongly vignetted regions of the focal plane,
which were truncated by CCD edges,
or which were otherwise affected by bad pixels in the detector.
We will explain below that none of such detections 
have been included in the catalogue
-- an alternative detection was available in essentially all these situations
because of the IPHAS field pair strategy --
and hence these internal warning flags do not appear
in the final source catalogue.

Finally, we note that basic information on the observing conditions
is included (\emph{fieldID}, \emph{fieldGrade}, \emph{night}, \emph{seeing}).
A table containing more detailed quality control information,
indexed by \emph{fieldID}, is made available on our website.

\subsection{Band-merging the detection tables}

The second step in compiling the source catalogue
is to merge the contemporaneous trios
of $r$, $i$, \ha\ detection tables
into multi-band field catalogues.
This required a positional matching procedure 
to link sources between the three bands
based on their position on the sky.
We used the \textsc{tmatchn} function 
of the \textsc{stilts} software for this purpose,
which allows rows from multiple tables to be matched \citep{Taylor2006}.
The result of the procedure is a band-merged catalogue
in which each row corresponds to a group of linked $r$/$i$/\ha\ detections
which satisfy a maximum matching distance criterion in a pair-wise sense.
Sources for which no counterpart was identified
are retained in the catalogue as single-band detections.

We employed a maximum matching distance of 1~arcsec,
which was chosen based on a trade-off between 
completeness and reliability.
On one hand, a matching distance larger than 1~arcsec 
was found to allow too many spurious and unrelated sources 
to be linked. 
On the other hand, a value smaller than 1~arcsec 
would pose problems for very faint sources 
with large centroiding errors, 
and would occasionally fail to link detections near CCD corners
where the astrometric solution can 
show systematic errors which exceed 0.5~arcsec.
The position offsets between the $r$ and $i$/\ha\ detections
have been included in the catalogue 
and can hence be tightened by the user if necessary
(columns \emph{iXi}, \emph{iEta}, \emph{haXi}, \emph{haEta}).
We note that also UKIDSS/GPS adopted 
a maximum matching distance of 1 arcsecond 
for similar reasons \citep{Hambly2008}.

The resulting band-merged catalogues were found
to be reliable for the vast majority of fields.
We warn that source blending and confusion is unavoidable
for faint objects in the Galactic Plane however;
in \S\ref{sec:discussion} we will show
that 19\% of the sources in our catalogue
are flagged as blended objects (column \emph{deblend})
and their band-merged data should be treated with care
because they may have fallen victim to source confusion
during the band-merging step.

\subsection{Selecting the primary detections}

We explained earlier that the survey contains repeat observations
of identical sources as a result of overlaps in the data-taking pattern.
Amongst all sources in the reliable magnitude range $13<r<19$,
we find that 65\% were detected twice and 25\% were detected three times or more.
Only 9\% were detected once.
Unsurprisingly, their spatial distribution traces
the inter-CCD gaps and footprint edges.

The principal aim of this data release is to provide 
reliable photometry at a single epoch,
and hence we have decided
to focus on providing the magnitudes
and coordinates using only the \emph{best-available} 
detection of each object -- 
hereafter referred to as the \emph{primary} detection.
Although overlapping fields could have been co-added 
to gain a small improvement in depth, 
we have decided against this for two reasons.
Firstly, combining the information from multiple epochs
would make the photometry of variable stars difficult to interpret.
Secondly, co-adding would cause the image quality to degrade towards the mean,
which is a draw-back for crowded fields.

Anyone interested in the alternative detections of a source
-- hereafter called the \emph{secondary} detections --
can nevertheless obtain this information in two ways.
To begin with, whenever a secondary detection was observed 
within 10 minutes of the primary,
the magnitudes of that secondary detection
have been included in the catalogue
(columns \emph{sourceID2}, \emph{fieldID2}, 
\emph{r2}, \emph{i2}, \emph{ha2},
\emph{rErr2}, \emph{iErr2}, \emph{haErr2}, \emph{errBits2}.
This is the case for 66\% of the sources brighter than $r<20$
due to the IPHAS field pair observing pattern.
In addition, the complete set of detection tables -- one for each exposure -- 
is made accessibly on our website to allow other uses of the data.
A user-friendly catalogue of secondary detections 
has not been compiled at present 
but may be part of a future data release.

The primary detection is defined as the entry in the 
set of band-merged field catalogues which provides 
the most reliable information for a unique source.
Primary detections have been selected using a so-called \emph{seaming} procedure
which has been adapted from the algorithm developed for the WSA\footnote{http://surveys.roe.ac.uk/wsa/dboverview.html\#merge}.
In brief, the first step of the procedure is to identify all the duplicate detections
by cross-matching the overlap regions of all field catalogues,
again using a maximum matching distance of 1\arcsec.
The duplicate detections for each unique source are then ranked according to
to (i) filter coverage, (ii) quality score (column \emph{errBits}),
and (iii) the average seeing of stars in the CCD frame rounded to 0.2~arcsec.
If this ranking scheme reveals multiple `winners' of identical quality,
then the one that was observed closest to the optical axis of the camera is chosen.


\subsection{Compiling the final source catalogue}

As the final step, the primary detections that have been
selected above are compiled
into the 98-column source catalogue
that is described in Appendix~\ref{app:columns}
and made available on line.
The entire list of sources naturally includes 
a significant number of spurious entries
as a result of the very sensitive detection levels
that are employed by the CASU pipeline by default.
To limit the size of the source catalogue,
we have decided to enforce three basic criteria
which must be met for a candidate source
to be included in the catalogue:
\begin{enumerate}
\item the source must have been detected at SNR$>5$ in at least
one of the bands, i.e. it is required that at least one of
\emph{rErr}, \emph{iErr} or \emph{haErr} is smaller
than 0.2 mag;
\item the shape of the source must not be an obvious
cosmic ray or noise artefact, i.e. we require
either \emph{pStar} or \emph{pGalaxy} to be
greater than 20\%;
\item the source must not have been detected in one of the strongly
vignetted corners of the detector, 
not have had any known bad pixels in the aperture,
and not have been on the edge of one of the CCDs,
i.e. we require the \emph{errBits} quality score
to be smaller than 64.
\end{enumerate}

A total of 219 million primary detections satisfied
the above criteria and have been included in the catalogue.
Amongst these, 158 million objects 
are detected in all three bands (72\%),
30 million are detected in two bands (14\%),
and 31 million entries are single-band detections (14\%).
Roughly half of the single-band detections were made in the $i$-band.
This is likely explained by the fact that the $i$-band is least
affected by interstellar extinction and can occasionally pick up
highly-reddened objects which are otherwise lost in $r$/\ha.

\section{Discussion}
\label{sec:discussion}

Having explained how the catalogue was generated,
we now offer an overview of its properties
by discussing  
(i) the recommended quality criteria,
(ii) the typical photometric uncertainties,
and (iii) the source densities and the frequency of source blending.

\subsection{Recommended quality criteria}
\label{sec:qualitycriteria}

Like any other photometric survey,
the majority of the objects in our catalogue
are faint sources observed near the detection limits:
55\% of the entries in the catalogue
are fainter than $r > 20$
and 18\% are even fainter than $r > 21$.
The measurements of such faint objects
are naturally prone to large
random and systematic uncertainties,
for example, an inaccurately subtracted background
will introduce a proportionally larger systematic error
in a faint object.
Most scientific applications will require a set of
quality criteria to be enforced for the purpose
of removing faint and low-quality objects.

The choice of quality criteria is often a complicated
trade-off between completeness on one hand
and accuracy on the other.
To aid users we have listed two sets of
recommended quality criteria 
in Tables~\ref{tab:reliable} and \ref{tab:veryreliable}.

Firstly, Table~\ref{tab:reliable} details
a set of minimum quality criteria
which should benefit most applications
which require reliable $r-i$ and $r-$\ha\ colours
without removing more than $\sim$80\%
of the sources brighter than $r < 19$.
The listed criteria are designed to 
(i) remove low-SNR sources, 
(ii) remove saturated sources,
and (iii) remove objects for which the 2.3~arcsec diameter
aperture magnitudes are inconsistent 
with the alternative 1.2~arcsec diameter aperture measurements.
The last criterion is a proxy
for identifying objects which are affected
by inaccurate background subtraction
or failed source deblending.
A total of 86 out of 219 million sources 
(39\%) satisfy all the criteria listed in Table~\ref{tab:reliable}
and are hereafter referred to as ``reliable''.
For convenience, the catalogue contains a boolean column
named \emph{reliable} which flags these objects
and makes their selection easy.

\begin{table*}
\vspace{2cm}

\begin{tabular}{p{8cm}lp{6cm}}
  \hline
  Quality criterion & Rows passed & Description \\
  \hline
  rErr\,$<$\,0.1 {\sc and} iErr\,$<$\,0.1 {\sc and} haErr\,$<$\,0.1 &
  109 million (50\%) &
  Require the photon noise to be less than 
  0.1 mag in all bands (i.e. SNR$>$10).
  This implicitly requires a detection in all three bands.  \\
  $r>13$ {\sc and} $i>12$ {\sc and} \ha\,$>12.5$ {\sc and not} \emph{saturated} &
  158 million (72\%) &
  The brightness must not exceed the nominal saturation limit
  and the peak pixel height must not exceed 55\,000 counts.
  Again, this implicitly requires a detection in all three bands.
  \\
  $|\mathrm{r}- \mathrm{rAperMag1}| 
  < 3\sqrt{\mathrm{rErr}^2 + \mathrm{rAperMag1Err}^2} 
  + 0.03$ &
  176 million (80\%) &
  Require the $r$ magnitude measured 
  in the default 2.3\arcsec\ diameter aperture
  to be consistent with the measurement 
  made in the smaller 1.2\arcsec\ aperture,
  albeit tolerating a 0.03 mag systematic error.
  This will reject sources for which the background
  subtraction or the deblending procedure
  was not performed reliably. \\
  $|\mathrm{i}- \mathrm{iAperMag1}| 
  < 3\sqrt{\mathrm{iErr}^2 + \mathrm{iAperMag1Err}^2} 
  + 0.03$ &
  183 million (84\%) &
  Same as above for $i$. \\
  $|\mathrm{ha}- \mathrm{haAperMag1}| < 
  3\sqrt{\mathrm{haErr}^2 + \mathrm{haAperMag1Err}^2} 
  + 0.03$ &
  158 million (72\%) &
  Same as above for \ha. \\
  \hline
  All of the above (flagged as {\bf\emph{reliable}}) &
  86 million (39\%) & \\
  \hline
\end{tabular}
\caption{Recommended minimum quality criteria 
for selecting objects with reliable colours 
from the IPHAS DR2 source catalogue. 
86~million entries in the catalogue (39\%)
satisfy all the criteria listed in this table.
For convenience, these have been flagged in the catalogue
using the column named \emph{reliable}.}
\label{tab:reliable}

\vspace{2cm}

\begin{tabular}{p{8cm}lp{6cm}}
  \hline
  Quality criterion & Rows passed & Description \\
  \hline
   reliable &
   86 million (39\%) &
   The object must satisfy the criteria listed in Table~\ref{tab:reliable}. \\
   
   pStar $>$ 0.9 &
   145 million (66\%) &
   % pStar > 0.89 is identical to mergedClass == -1
   The object must appears as a perfect point source,
   as inferred from comparing its Point Spread Function (PSF)
   with the average PSF measured in the same CCD. \\
   
   {\sc not} \emph{deblend} &
   177 million (81\%) &
   The source must appear as a single, unconfused object. \\
   
   {\sc not} \emph{brightNeighb} &
   216 million (99\%) &
   There is no star brighter than $V < 4$ within 10 arcmin, 
   or brighter than $V < 7$ within 5 arcmin.
   Such very bright stars cause scattered light and diffraction spikes,
   which may add systematic errors to the photometry
   or even trigger spurious detections. \\  
  \hline
  
  All of the above (flagged as {\bf\emph{veryReliable}}) &
  59 million (27\%) & \\
  \hline
\end{tabular}
\caption{Additional quality criteria which are recommended
for applications which require very reliable colours
at the expense of completeness. 
For convenience, the sources which satisfy the criteria listed
in this table have been flagged in the catalogue
using the column named \emph{veryReliable}.}
\label{tab:veryreliable}

\vspace{2cm}
\end{table*}

For applications which require
an even higher standard of reliability,
a further set of additional quality criteria
are listed in Table~\ref{tab:veryreliable}.
These criteria are designed to ensure that
(i) the object appeared as a perfect point source,
(ii) the object was not blended with a nearby neighbour,
and (ii) the object was not located near a very bright star.
59 million sources (27\%) satisfy
these additional criteria 
and are hereafter referred to as ``very reliable''.
Again, the catalogue contains a boolean column
named \emph{veryReliable} which flags these objects.

Figure~\ref{fig:magdist} compares the $r$-band magnitude
distribution for objects with and without the 
\emph{reliable} and \emph{veryReliable} criteria applied. 
We find that 81\% of the sources 
in the magnitude range $13 < r < 19$
are considered \emph{reliable},
which drops to 72\% in the range $19 < r < 20$
and 9\% at $r>20$.
In contrast, only 54\% of the sources 
in the magnitude range $13 < r < 19$
are considered \emph{veryReliable}.
The stricter criteria filter out a lot
of objects at early Galactic longitudes 
where source blending is a common problem
(we will demonstrate this in \S\ref{sec:densities}).
The \emph{veryReliable} flag should hence
only be used in applications which require very reliable photometry
at the expense of completeness,
which might be the case for e.g. spectroscopic target selection.

\begin{figure}
    \includegraphics[width=0.5\textwidth]{./plots/magdist/magdist-r.pdf} 
    \caption{r-band magnitude distribution for all objects in the catalogue 
    (light grey), for objects flagged as \emph{reliable} 
    according to the criteria set out in Table~\ref{tab:reliable} (grey),
    and for objects flagged as \emph{veryReliable} 
    following Table~\ref{tab:veryreliable} (dark grey).
    The magnitude distributions for $i$ and \ha\
    look identical, apart from being shifted
    by about 1 and 0.5 mag towards brighter magnitudes,
    respectively.}
    \label{fig:magdist}
\end{figure}

It is easy to see how the quality criteria
may be adapted to be more tolerant.
For example, by raising the allowed photometric uncertainties
from 0.1 mag to 0.2 mag one can retrieve 42 million candidate sources.

\subsection{Photometric uncertainties}

Figure~\ref{fig:uncertainties} shows the mean photometric
uncertainties as a function of magnitude for each band.
We find that the uncertainties typically
reach 0.1 mag near 20.5 in $r$ 
and 19.5 in $i$/\ha when the default 2.3\arcsec\ aperture is used.
At this point we note that the average colour
of objects in the survey is
$1.06\pm0.12$ for ($r$-$i$) and $0.44\pm0.03$ for ($r$-\ha).
The better depth of $r$ is hence compensated
by the fact that stars tend to have 
brighter magnitudes in $i$ and \ha.

\begin{figure}
    \includegraphics[width=0.5\textwidth]{./plots/errors/uncertainties.pdf} 
    \caption{Mean photometric uncertainties
             for $r$ (top), $i$ (middle) and \ha\ (bottom).
             Data points shown are the average values of
             columns \emph{rErr}, \emph{iErr} and \emph{haErr}
             in the catalogue, 
             and the errorbars show the standard deviations.
             The dashed and solid lines indicate 
             the 10$\sigma$ and 5$\sigma$ limits, respectively.
             These uncertainties are based only on the (Poissonian)
             photon noise and hence this figure does not show
             systematic or calibration uncertainties.}
    \label{fig:uncertainties}
\end{figure}

The uncertainties shown in Fig.~\ref{fig:uncertainties}
are the random errors based on the expected Poissonian photon noise.
Systematics, such as calibration and deblending errors,
are not included.
To appraise the extent to which our photometry is affected
by such systematics, we can exploit the
secondary measurements which were made as part of the
field-pair observing strategy and are available for 51\%
of the sources in the catalogue.

Figure~\ref{fig:pairmag} shows the mean residuals between
the primary and secondary magnitudes
-- i.e. the average difference between catalogue columns \emph{r}-\emph{r2},
\emph{i}-\emph{i2}, \emph{ha}-\emph{ha2} -- as a function of magnitude.
We find that sources across the magnitude range 
$13 < r < 17$ are consistent at the level of 5\%
(i.e. $\sigma_{r-r2} \le 0.05$ mag),
with the best repeatability
reached at $r=14$ ($\sigma_{r-r2}~=~0.041$~mag).
We draw attention to the fact that brighter stars
tend to show significantly larger residuals
-- e.g. $\sigma_{r-r2}~=~0.14$~mag at $r=12$ --
which is due to saturation effects.
At the faint end we find residuals which
show significantly more scatter than would be expected
from photon noise alone, that is,
the effects of source blending and background subtraction
appear to dominate from $\sim$20th magnitude onwards.

\begin{figure*}
    \vspace{1cm}
    \includegraphics[width=0.5\textwidth]{./plots/pairmag/pairmag.pdf} 
    \caption{Photometric repeatability illustrated by plotting
             the mean residuals between all the primary and secondary detections
             in the catalogue as a function of magnitude.
             The best photometric repeatability is reached at $r=14$
             with $\sigma_{r-r2}~=~0.041$~mag.
             Note that bright stars at $r<13$ and $i<12$ 
             show increasing uncertainties due to saturation effects.}
    \label{fig:pairmag}
    \vspace{1cm}
    \includegraphics[width=0.5\textwidth]{./plots/pairmag/pairmag-reliable.pdf} 
    \caption{Same as Figure~\ref{fig:pairmag},
    except that only the subset of sources
    flagged as \emph{veryReliable} are now included.
    We find that applying the quality criteria
    has improved the photometric repeatability significantly.
    The best repeatability is again reached at $r=14$
    but has reduced to $\sigma=0.028$ mag.
    The quality criteria have also been successful
    at removing objects with large systematics at the bright
    and faint ends.}
    \label{fig:pairmag_reliable}
\end{figure*}

In Figure~\ref{fig:pairmag_reliable} we show 
a similar comparison of the primary and secondary detections,
but this time we have only include sources which are flagged
as \emph{veryReliable} in the catalogue
(i.e. not saturated, not confused, not near bright stars, etc.)
We find that the average residuals are significantly better
for this subset of the catalogue. Sources across the magnitude range 
$13 < r < 17$ are consistent at the level of 0.03 mag,
and the best repeatability is again reached at $r=14$
with $\sigma_{r-r2}~=~0.028$~mag.
We conclude that the \emph{veryReliable} quality criteria are effective
in reducing the systematic errors to
the same level as the accuracy of the
global photometric calibration (cf. \S\ref{sec:calibration}).
Moreover, the large systematics at the bright and faint end
have disappeared.

\subsection{Source densities and blending problems}
\label{sec:densities}

The mean source density as a function of Galactic longitude
is shown in Figure~\ref{fig:density} (thick blue line).
The densities were computed by counting the 
number of sources in $1^\circ$-wide longitude bins
across which covered the entire latitude range $-5^\circ<b<+5^\circ$.
Unsurprisingly, we find the average source densities to increase
towards the Galactic centre.
For example, the average source density near $l\simeq 30^\circ$
is roughly 30\,000 objects per square degree,
which is five times more than the density
found near $l\simeq 180^\circ$.

In addition to the global trend, 
there are significant variations in the source density on smaller scales.
For example,  we find a significant drop near the constellations 
of Cygnus ($l\simeq 80$) and Aquila ($l\simeq 40$),
which are regions known to be affected
by high levels of foreground extinction.
Dark clouds are visible towards these constellation by eye,
and they are often referred to as ``the Great Rift''.

We warn however that the densities reported here 
have not been corrected for survey completeness
or differences in the observing conditions across the survey.
For example, the dip in the density near $l\simeq140^\circ$
is an artificial feature caused by gaps
in the footprint coverage (which are apparent in Fig.~\ref{fig:footprint}).
In a forthcoming paper,
we aim to calibrate the IPHAS-based source densities
by injecting artificial stars into the IPHAS images
and measuring their recovery rate (Farnhill et al., in preparation).
Indeed IPHAS has the potential to offer
calibrated, two-dimensional stellar density maps
which can be used to constrain detailed models of our Galaxy,
but it is beyond the scope of the present work.

\begin{figure*}
    \includegraphics[width=\textwidth]{./plots/density/density.pdf} 
    \caption{Mean number density of sources in the catalogue 
    as a function of Galactic longitude, 
    with and without blended sources included. 
    The densities shown were computed by counting the sources 
    at each longitude between $-5^\circ<b<+5^\circ$ (upper blue line).
    We also show the densities based on only counting those sources 
    for which the \emph{deblend} flag is {\sc false}, 
    i.e. unconfused sources for which the CASU pipeline did not have to apply 
    a deblending procedure (lower red line). 
    }
    \label{fig:density}
\end{figure*}

In Figure~\ref{fig:density} we also shows the density
of non-blended sources (thin red line).
These are sources for which the \emph{deblend} flag is {\sc false},
i.e. sources for which the CASU pipeline did not have to apply 
a deblending procedure to separate the flux
originating from two or more overlapping objects.
We find a strong correlation between the source density
and the fraction of sources affected by source blending.
For example, only NN\% of the sources are blended
at $l>90^\circ$, whereas NN\% are blended at $l<90^\circ$.

As we explained earlier, blended sources must be used with caution.
Firstly, the deblending-procedure crucially depends
on the local PSF being measured accurately. 
Secondly, blended sources may are likely candidates to have fallen victim
to source confusion during the band-merging procedure.
In future work we hope to investigate the use of
more advanced PSF-fitting routines
in which sources are measured simultaneously across all bands,
perhaps guided by an external list of sources provided
by near-infrared surveys or Gaia data.

\section{Demonstration}
\label{sec:demonstration}

We conclude this paper by demonstrating how the unique
$r$-$i$/$r$-\ha\ colour-colour diagram offered by this catalogue
can readily be used to
(i) characterise the extinction regime at different sightlines, and
(ii) identify \ha\ emission-line objects.

\subsection{Colour-colour and colour-magnitude diagrams}

The survey's unique $r$-\ha\ colour,
when combined with $r$-$i$,
has been shown to provide simultaneous constraints 
on intrinsic stellar colour and interstellar extinction \citep{Drew2008}. 
That is, the main sequence in the $r$-$i$/$r$-\ha\ diagram
runs in a direction that is different from the reddening vector,
because the $r$-\ha\ colour tends to act as a coarse proxy for spectral type
and is less sensitive to reddening than $r$-$i$.
As a result, the distribution of a stellar population
in the IPHAS colour-colour diagram
can offer a handle on the properties and extinction regime
along a line of sight.

This is demonstrated in Figures \ref{fig:l180}, \ref{fig:l45} \& \ref{fig:l30},
where we present three sets of IPHAS colour/magnitude diagrams
towards three distinct sightlines
located at Galactic longitudes $180^\circ$, $45^\circ$ and $30^\circ$,
respectively.
Each figure contains all the sources which are flagged as \emph{veryReliable}
and are located in a region of one square degree 
centred on the coordinates indicated in the diagram
(i.e. within a radius of $0.564^\circ$ from the indicated sightline).
For clarity, we have imposed the additional criterion
for photometric uncertainties to be smaller than 0.05 mag in each band
(corresponding to a magnitude limit near $\sim$19th magnitude, effectively).

Each of the diagrams reveals a well-defined locus,
which demonstrates the health of the catalogue and the global calibration
for investigating stellar populations across wide areas.
We have annotated the colour-colour diagrams
by showing the position of the unreddened main sequence (thin solid line),
the unreddened giant branch (thick solid line)
and the reddening track for an A0V-type star (dashed line)
-- all three are based on the \cite{Pickles1998} library of empirical spectra
tabulated for IPHAS by \cite{Drew2005}.
In the colour-magnitude diagrams we only show the reddening vector
together with the unreddened 1~Gyr isochrone due to \cite{Bressan2012},
which are made available for the IPHAS filter system through a
popular on line tool hosted by the Observatory of Padova (http://stev.oapd.inaf.it/cmd).
The isochrone and reddening vector has been placed
at an arbitrary distance of 2~kpc.

Each of the sightlines reveals a stellar population
with distinct characteristics.
Firstly, towards the Galactic anti-centre at $l=180^\circ$ (Fig.~\ref{fig:l180}) 
we find a population dominated by lowly-reddened 
main sequence stars and an apparent absence of reddened giants.
%consistent with the sightline extinction of XX from Schlegel (cite). 
In contrast, closer towards the galactic
centre at $l=45^\circ$ (Fig.~\ref{fig:l45})
we find a wealth of reddened objects
which appear to be separated into a lowly
and a highly reddened component,
perhaps revealing two distinct parts of the Galaxy.
Finally, in one of our earliest sightlines at $l=30^\circ$
we find a very high number of extremely reddened giants
in addition to an unreddened population
of foreground dwarfs.

The number density of stars in the colour-colour
and colour-magnitude space
can be compared against population synthesis models
to create three-dimensional maps of the extinction
across several kpc \citep{Sale2010,Sale2012}.
Such an extinction map based on our catalogue
is to be released in a separate paper
that accompanies this data release (Sale et al., in preparation).

\begin{figure*}
    \begin{minipage}[b]{\linewidth}
        \includegraphics[width=0.5\textwidth]{./plots/ccd-180-3.pdf} 
        \includegraphics[width=0.5\textwidth]{./plots/cmd-180-3.pdf}
    \end{minipage}
    \caption{Colour-colour and colour-magnitude diagram (left and right panel)
    showing sources flagged as \emph{veryReliable}
    located in an area of one square degree
    centred near the Galactic anti-centre 
    at $(l,b)=(180^\circ,+3^\circ)$.
    The colour-colour diagram shows the
    position of the main sequence (thin solid line),
    giant stars (thick solid line)
    and the reddening track for an A0V-type star (dashed line)
    based on the \citet{Pickles1998} library of empirical spectra.
    The colour-magnitude only show the reddening vector
    along with the unreddened 1~Gyr isochrone due to \citet{Bressan2012},
    which has been placed at an arbitrary distance of 2~kpc for reference.
    This is one of the least reddened sightlines
    in the survey %(Av XX, Schlegel et al. XXXX)
    and hence the observed stellar population appears to be dominated 
    by lowly reddened main sequence stars.}
    \label{fig:l180}
    \begin{minipage}[b]{\linewidth}
        \includegraphics[width=0.5\textwidth]{./plots/ccd-45-2.pdf}
        \includegraphics[width=0.5\textwidth]{./plots/cmd-45-2.pdf}
    \end{minipage}
    \caption{Same as above for $(l,b)=(45^\circ,+2^\circ)$,
    which is one of the highest-density sightlines in the survey,
    revealing a population of stars with a reddening distribution
    that appears to be bi-model.}
    \label{fig:l45}
    \begin{minipage}[b]{\linewidth}
        \includegraphics[width=0.5\textwidth]{./plots/ccd-30-0.pdf}
        \includegraphics[width=0.5\textwidth]{./plots/cmd-30-0.pdf} 
    \end{minipage}
    \caption{Same as above for $(l,b)=(30^\circ,0^\circ)$.
    This is one of the most reddened sightlines in the survey.
    %(XXX, Schlegel et al XXXX).
    }
    \label{fig:l30}
\end{figure*}

\subsection{Identifying \ha\ emission-line objects}

A primary motivation for carrying
out the survey 
was to enable the discovery of 
new emission-line objects across the Galactic Plane.
\ha\ in emission is a well-known tracer
for stars in the short-lived pre- or
post-main sequence stages of their evolution,
and hence IPHAS aims to allow larger, deeper
and more statistically robust samples of such rare objects
to be established.
The survey-wide identification and analysis 
of such stars is beyond the scope of the present work,
but in this section we demonstrate how the
catalogue may be used for this purpose.

An initial list of candidate \ha-emitters
based on the first IPHAS data release has previously
been presented by \cite{Witham2008}. 
Because no global uniform calibration was available
at the time, \citeauthor{Witham2008} employed 
a sigma-clipping technique to select objects with
large, outlying $r$-\ha\ colours.
In contrast, the new catalogue
allows objects to be picked out
from the $r$-$i$/$r$-\ha\ colour-colour diagram
using model-based colour criteria
rather than a statistical procedure.
In what follows we demonstrate this ability 
by selecting candidate emission-line objects
towards a small region in the sky.

The target of our demonstration is Sh 2-82:
a 5~arcmin-wide H{\sc ii} region located near $(l,b)=(53.55^\circ, 0.00^\circ)$
in the constellation of Sagitta.
Nicknamed by amateur astronomers as the ``Little Cocoon Nebula'',
Sh 2-82 is ionised by 
the $\sim$10th magnitude star HD\,231616
% Georgelin1973 claims B0III, others claim B0V
with spectral type B0V/III
\citep{Georgelin1973,Mayer1973,Hunter1990}.
The ionising star has been placed at a likely distance of 1.5-1.7 kpc
based on the photometric parallax
\citep{Mayer1973,Lahulla1985,Hunter1990}.

Figure~\ref{fig:mosaic_iphas} shows a 20-by-15 arcmin
colour mosaic centred on Sh 2-82,
composed of our \ha\ (red channel),
$r$ (green channel),
and $i$ (blue channel) images.
The ionising star can be seen as the bright object
near the centre of the H{\sc ii} region,
which is surrounded by a faint reflection nebula
and several dark cloud filaments.
For comparison, Figure~\ref{fig:mosaic_spitzer} shows
a mosaic of the identical region 
as seen by the Spitzer Space Telescope
in the mid-infrared. The Spitzer image
reveals a bubble-shaped structure of warm dust
which surrounds the entire H{\sc ii} region.

\begin{figure*}
    \begin{minipage}[b]{0.8\linewidth}
        \includegraphics[width=\textwidth]{./plots/mosaic/sh2-82-iphas.pdf} 
    \end{minipage}
\caption{IPHAS-based mosaic of H{\sc ii} region Sh 2-82,
composed of \ha\ (red channel), $r$ (green channel) and $i$ (blue channel). Yellow triangles show the position of candidate \ha-emitters
which have been selected from the colour-colour diagram
in Figure~\ref{fig:emitters}. Note that the H{\sc ii} region is surrounded by a faint blue/green reflection nebula
and dark cloud filaments.}
\label{fig:mosaic_iphas}
    \begin{minipage}[b]{0.8\linewidth}
        \includegraphics[width=\textwidth]{./plots/mosaic/sh2-82-spitzer.pdf} 
    \end{minipage}
    \caption{Star-forming region Sh 2-82 as seen in the mid-infrared
    by the Spitzer Space Telescope. The mosaic is composed of the 24\,\micron\ (red), 8.0\,\micron\ (green) and 4.5\,\micron\ (blue) bands.
    The image reveals a bubble-shaped structure which surrounds the {\sc Hii} region that is seen in the IPHAS mosaic of the same region (Figure~\ref{fig:mosaic_iphas}).}
    \label{fig:mosaic_spitzer}
\end{figure*}

Figure~\ref{fig:emitters} presents
the IPHAS colour-colour diagram for the region covered by the
mosaic images.
Gray circles show all objects in the region
which are brighter than $r<20$
and have been flagged as \emph{reliable}
in our catalogue.
The diagram also shows the unreddened main sequence (solid line)
and the expected position of unredded main-sequence stars
with \ha\ in emission at a strength of EW=$-10{\rm \AA}$ (dashed line),
taken from the colour simulations due to \citet{Barentsen2011a}.
Six stars are found to lie confidently above the 
dashed line at the level of $3\sigma$ 
(i.e. the distance between the dashed line is larger than
three times the uncertainty on the $r$-\ha\ colour).
These reliable candidate \ha-emitters
are marked by red triangles in the colour-colour diagram
and their photometry is detailed in Appendix~\ref{app:emitters}.

\begin{figure}
  \includegraphics[width=0.45\textwidth]{./plots/sh2-82-ccd.pdf}
    \caption{$r$-$i$/$r$-\ha\ diagram for the rectangular region of 
    20-by-15 arcmin centred on the H{\sc ii} region Sh 2-82,
    which is the area shown in Figure~\ref{fig:mosaic_iphas}.    
    The diagram shows all objects in the catalogue
    which have been flagged as \emph{reliable} and are brighter
    than $r<20$ (grey circles).
    The unreddened main sequence is indicated by a solid line,
    while the main sequence for stars with an \ha\ emission line
    strength of $-10\,\rm{\AA}$ EW is indicated by a dashed line,
    following the colour simulations due to \citet{Barentsen2011a}.
    Red triangles indicate objects which have been identified as
    as highly likely \ha-emitters (see text).}
    \label{fig:emitters}
\end{figure}

The spatial distribution of our six candidate emission-line objects
is marked by yellow triangles in the colour mosaic (Fig.~\ref{fig:mosaic_iphas}).
They are likely to be genuine young stars for two reasons.
Firstly, two of our candidates have recently been identified as likely
candidate Young Stellar Objects (YSO)
in an investigation of the region by \cite{Yu2012}.
In their study, the authors used 2MASS and Spitzer data
to select likely young stars by selecting objects with
circumstellar disks based on the infrared colour excess.
Secondly, we find that the four remaining objects are 
also detected in the Spitzer-based image,
although their colours are less extreme
than those identified as likely YSOs by \citeauthor{Yu2012}.

Prior to IPHAS, this region was essentially unstudied
at faint magnitudes in visible light.
\cite{Lahulla1985} reported magnitudes for 8 stars in the optical at 
$V < 15$. In contrast, the IPHAS catalogue offers photometry
for NN stars in the region down to $r<20$.
This demonstrates the ability of IPHAS to providing complimentary
for the wealth of poorly-studied star-forming regions
at low Galactic latitudes,
which have been unveiled in recent years
by the observed wealth of star-forming ``bubbles''
at mid-infrared wave%\cite{Bica2003} was the first to identify a
%near-infrared cluster
%associated with this region using near-infrared 2MASS data,
%and only recently has this cluster been confirmed
%in a short investigation which combined 2MASS
%and Spitzer data \cite{Yu2012}.
%Many of the young stars identified by \citeauthor{Yu2012}
%can be seen as pink and red stars in Fig.~\ref{fig:mosaic_spitzer}.lengths \cite[e.g.][]{Churchwell2006,Simpson2012}.
%In fact the cluster of young stars associated with Sh 2-82
%was only first reported by \cite{Bica2003},
%remained unstudied until the study by \cite{Yu2012}.

%Amongst the six emitters, N are detected at SNR X
%in the Spitzer X micron band. 

%This demonstrates how IPHAS is providing complimentary
%data for some of the thousands of star-forming ``bubbles'' have been unveiled at mid-infrared wavelengths \cite[e.g.][]{Churchwell2006,Simpson2012}.
%Optically-unveiled members of such star-forming regions
%may help to constrain extinctions and distances towards such regions \citep{Barentsen2013}.
%Moreover, in \cite{Barentsen2011a} we showed
%how the history of star formation in a region,
%and the possible triggered nature, may be discussed.
%This is particularly true 

%TODO: annotate -10 EW text on plot?
%TODO: mention Gaia, e.g. IPHAS for transient identification?

%\subsection{Caveats and lessons learnt}
%\cite{Bica2003} was the first to identify a
%near-infrared cluster
%associated with this region using near-infrared 2MASS data,
%and only recently has this cluster been confirmed
%in a short investigation which combined 2MASS
%and Spitzer data \cite{Yu2012}.
%Many of the young stars identified by \citeauthor{Yu2012}
%can be seen as pink and red stars in Fig.~\ref{fig:mosaic_spitzer}.

\section{Conclusions and future work}
\label{sec:conclusions}

A new data release for the IPHAS survey was presented,
taking the coverage up to over 90\% of the Northern Galactic Plane 
at $|b|<5^\circ$
and providing a uniform photometric calibration
for the first time.
We explained the data reduction and quality control procedures that
were applied, described and tested the new global photometric calibration,
and detailed the construction of the user-friendly source catalogue.

The observations included in this release
were found to achieve a median seeing of 1.1 arcsec
and a $5\sigma-$depth of $r=21.2\pm 0.5$, $i=20.0\pm 0.3$, \ha$=20.3\pm 0.3$.
The global calibration and photometric repeatability
is accurate at the level of $\sigma=0.03$ mag,
providing a significant improvement over the 
previous data release.
The source catalogue provides the best-available
single-epoch astrometry and photometry
for 219~million unique sources.

The data-taking strategy developed for IPHAS
have since been reapplied to carry out a companion INT/WFC survey called UVEX
in U/$g$/$r$ \citep{Groot2009},
and also southern counterpart to IPHAS and UVEX 
is being carried out in $u$/$g$/$r$/$i$/\ha\ 
called VPHAS+ (Drew et al, in press).
We hope to re-use the experience gained by this data release
to create similar releases for these companion surveys.

In future work, we aim to draw upon the PanSTARRS photometric
survey to further improve the accuracy of our global calibration.
We will also aim to correct the photometry for the
radial field distortions.
%We may also produce a catalogue which details all detections.

%TODO: mention Gaia?

\section*{Data access and source code}
\label{sec:dataaccess}

The catalogue is made available through the Vizier
catalogue tool (http://vizier.u-strasbg.fr),
where it is known as the ``IPHAS DR2 Source Catalogue''
(catalogue ID ???).
The full catalogue can also be downloaded in its entirety
from the IPHAS website (www.iphas.org) as a collection 
of binary FITS tables which comprise 50\,GB,
which is accompanied by a script
to ingest the data into a PostgreSQL database.
Our website also provides access to the pipeline-processed
imaging data, which we have updated to include
the re-calibrated DR2 zeropoint in the image headers.

The source code that was used to generate
the catalogue is available at
https://github.com/barentsen/iphas-dr2

\section*{Acknowledgments}

The IPHAS survey was carried out 
at the Isaac Newton Telescope (INT).
The INT is operated on the island of La Palma
by the Isaac Newton Group
in the Spanish Observatorio del Roque de los Muchachos
of the Instituto de Astrofisica de Canarias.
All data were processed 
by the Cambridge Astronomical Survey Unit,
at the Institute of Astronomy in Cambridge.

Preparation of the catalogue was eased greatly
by a number of software packages,
including the Python modules
AstroPy \citep{Astropy},
APLpy, NumPy and SciPy,
the PostgreSQL database software,
the TOPCAT and STILTS packages \citep{Taylor2005,Taylor2006},
and the Montage software maintained by NASA/IPAC.
We also made use of the SIMBAD, Vizier and Aladin \citep{Aladin} tools operated at CDS, Strasbourg, France.

This research made extensive use of
several complementary photometric surveys.
Our global calibration was aided
by the AAVSO Photometric All-Sky Survey (APASS),
funded by the Robert Martin Ayers Sciences Fund.
The calibration was tested against the
Sloan Digitized Sky Survey (SDSS),
funded by the Alfred P. Sloan Foundation, the Participating Institutions, the National Science Foundation, the U.S. Department of Energy, the National Aeronautics and Space Administration, the Japanese Monbukagakusho, the Max Planck Society, and the Higher Education Funding Council for England.
The astrometric pipeline reduction made
significant use of the Two Micron All Sky Survey (2MASS),
which is a joint project 
of the University of Massachusetts
and the Infrared Processing and Analysis Center/
California Institute of Technology,
funded by NASA and the NSF.

GB and JED acknowledge the support of a grant
from the Science \& Technology Facilities Council
of the UK (STFC, ref ST/J001335/1).
HJF is in receipt of an STFC postgraduate studentship.

\bibliographystyle{mn2e}
\bibliography{dr2}

\appendix
\input{app_columns.tex}

\section{Candidate emission-line objects towards Sh 2-82}
\label{app:emitters}
\begin{table*}
    \begin{tabular}{lccc}
    \toprule
    Name & $r$ & $i$ & \ha  \\
    \midrule
IPHAS2 J192954.40+181026.1& $17.69\pm0.01$ & $16.12\pm0.01$ & $16.19\pm0.01$ \\
IPHAS2 J193011.01+182051.2& $18.55\pm0.02$ & $16.95\pm0.02$ & $17.31\pm0.02$ \\
IPHAS2 J193021.52+181954.5& $19.72\pm0.05$ & $17.94\pm0.03$ & $18.47\pm0.04$ \\
IPHAS2 J193024.45+181938.3& $19.31\pm0.04$ & $17.55\pm0.02$ & $17.99\pm0.03$ \\
IPHAS2 J193033.00+181609.3& $18.25\pm0.01$ & $16.91\pm0.01$ & $16.92\pm0.01$ \\
IPHAS2 J193042.48+182317.4& $19.96\pm0.03$ & $18.11\pm0.03$ & $18.48\pm0.03$ \\
    \bottomrule
    \end{tabular}
    \caption{Candidate \ha-emitters towards Sh 2-82.}
    \label{tbl:emitters}
\end{table*}

\label{lastpage}

\end{document}

\documentclass[12pt]{article}
\usepackage{graphicx}
\usepackage{subfigure}
\usepackage[hyphens]{url}
\usepackage[multiple]{footmisc}
\usepackage{longtable}
\usepackage[top=1.5in, bottom=1.0in, left=0.5in, right=0.5in]{geometry}

\newcommand{\tbd}{{\tiny \textbf{TBD}}}

\title{IPHAS DR2: Table specification}
\date{\today}
\author{Geert Barentsen}

\begin{document}

\maketitle

This document describes the proposed set of tables and columns which are to be 
included in IPHAS Data Release 2. In brief, it is proposed to create four 
tables:
\begin{itemize}
\item \textbf{iphasDetection}: the complete collection of all objects detected 
in those IPHAS exposures which pass the essential quality criteria. Each row 
represents a single-epoch, single-band detection, and hence this table 
contains duplicate detections of unique objects.
\item \textbf{iphasSource}: band-merged version of the detections table, in 
which detections have been cross-matched on a field-by-field basis. Unique 
sources may have duplicate entries in this table, but the `best' entry for 
each unique source will be flagged. 
\item \textbf{iphasExposure}: meta-data on the observing conditions for each 
run.
\item \textbf{iphasField}: meta-data on the observing conditions, based on 
iphasExposure but collated on a field-by-field basis with extra information added (cf. the existing iphas-qc.fits table).
\end{itemize}

On top of these four tables we may define `views' which define subsets of 
certain rows and columns for the purpose of being user-friendly. For example, 
there might be a view called \textbf{iphasPointSource} which consists of those 
entries from iphasSource which are marked `primary detections', hence avoiding duplicates. In what follows we concentrate on defining the four main tables however.

It is important to note that the naming conventions and column descriptions 
used below are largely based on two existing tables in the UKIDSS/GPS survey, 
namely `gpsDetection' and `gpsSource' (see the WFAU website for the 
definitions of these tables\footnote{\url{http://surveys.roe.ac.uk/wsa/www/WSA_TABLE_gpsSourceSchema.html}}
\footnote{\url{http://surveys.roe.ac.uk/wsa/www/WSA_TABLE_gpsDetectionSchema.html}}). 
There are several reasons to adopt the UKIDSS format:
\begin{itemize}
\item a consistent format will encourage UKIDSS users to adopt IPHAS;
\item UKIDSS have an existing algorithm in place to merge the single-band 
morphological classifications into `mergedClass'/`mergedClassStat' columns, as 
explained in their schema browser
\footnote{\url{http://surveys.roe.ac.uk/wsa/www/gloss_m.html#gpssource_mergedclass}}
\footnote{\url{http://surveys.roe.ac.uk/wsa/www/gloss_m.html#gpssource_mergedclassstat}};
\item UKIDSS have an existing method to flag duplicate detections and mark the 
best one\footnote{\url{http://surveys.roe.ac.uk/wsa/www/gloss_p.html#gpssource_priorsec}}.
\end{itemize}


\newpage
\section{Table definitions}
\subsection{iphasDetection}

\begin{center}
\begin{longtable}{llclp{10cm}}
\caption[short]{Definition of the \textbf{iphasDetection} table.} \label{iphasSource} \\

\hline \textbf{Name} & \textbf{Type} & \textbf{Length} & \textbf{Unit} & \textbf{Description} \\ \hline
\endfirsthead

\hline \textbf{Name} & \textbf{Type} & \textbf{Length} & \textbf{Unit} & \textbf{Description} \\ \hline
\endhead

\hline \multicolumn{5}{r}{{\it Continued on next page}} \\ 
\endfoot

\hline \hline
\endlastfoot

detectionID & int & 8 & & Unique identification number formed by concatenating runID (6 or 7 digits), ccd (1 digit), seqNum (6 digits). \\
runID & int & 4 & & Original identifier of the run (i.e. exposure) in the observing logs of the Isaac Newton Telescope (INT). 
References to the iphasExposure table. \\
ccd & int & 1 & & Extension number of the CCD of the Wide-Field Camera (WFC). \\
seqNum & int & 4 & & Unique running number of this detection (unique withing the given runID/ccd combination, assigned by the image detection pipeline). \\
band & char & 2 & & Filter used. One of `r', `i', `ha'. \\ 
x & real & 4 & pixel & X coordinate of the detection in the CCD reference frame. \\
y & real & 4 & pixel & Y coordinate of the detection in the CCD reference frame. \\
ra & real & 8 & degrees & J2000 right ascension with respect to the 2MASS PSC reference frame (which is consistent with ICRS to within 0.1 arcsec). \\
dec & real & 8 & degrees & J2000 declination with respect to the 2MASS PSC reference frame (which is consistent with ICRS to within 0.1 arcsec). \\
posErr & real & 4 & arcsec & Astrometric fit error (RMS). \\

gauSig & real & 4 & pixels & RMS of axes of ellipse fit, i.e. sqrt(a**2 + b**2). \\
ell & real & 4 & & Ellipticity (defined as 1-b/a, where a/b=semi-major/semi-minor axes).  \\
pa & real & 4 & & Position angle of ellipse major axis with respect to the x-axis. \\

peakMag & real & 4 & mag & Aperture corrected magnitude in the Vega system 
(based on the peak pixel height.) \\
peakMagErr & real & 4 & mag & Uncertainty in peakMag. Does not include systematics. \\ 
aperMag1 & real & 4 & mag & Aperture corrected magnitude in the Vega system (1.17 arcsec aperture diameter, i.e. 1/2 x rcore radius). \\ % 1/2 x 7 pixels
aperMag1err & real & 4 & mag & Uncertainty in aperMag1. Does not include systematics. \\ 
aperMag2 & real & 4 & mag & Aperture corrected magnitude in the Vega system (2.33 arcsec aperture diameter, i.e. rcore radius). \\ % 7 pixels
aperMag2err & real & 4 & mag & Uncertainty in aperMag2. Does not include systematics. \\ 
aperMag3 & real & 4 & mag & Aperture corrected magnitude in the Vega system (3.30 arcsec aperture diameter, i.e. sqrt(2) x rcore radius). \\ % sqrt(2) x 7 pixels
aperMag3err & real & 4 & mag & Uncertainty in aperMag3. Does not include systematics. \\ 
aperMag4 & real & 4 & mag & Aperture corrected magnitude in the Vega system (4.67 arcsec aperture diameter, i.e. 2 x rcore radius). \\ % 2 x 7 pixels
aperMag4err & real & 4 & mag & Uncertainty in aperMag4. Does not include systematics. \\ 
aperMag5 & real & 4 & mag & Aperture corrected magnitude in the Vega system (6.60 arcsec aperture diameter, i.e. 2 sqrt(2) x rcore radius). \\ % 2 x sqrt(2) x 7 pixels
aperMag5err & real & 4 & mag & Uncertainty in aperMag5. Does not include systematics. \\ 
sky & real & 4 & {\sc ADU} & Local interpolated sky level from background tracker. \\
skyVar & real & 4 & {\sc ADU} & Local estimate of variation in sky level around 
image. \\
class & int & 2 & & Flag indicating the most probable morphological classification (1=galaxy, 0=noise, -1=star, -2=probableStar, -3=probableGalaxy, -9=saturated).\\
classStat & real & 4 & & N(0,1) stellarness-of-profile statistic. \\
badPix & real & 4 &  & Number of bad pixels. \\
deblend & bool & 1 &  & True if the object had to be deblended, i.e. if it was part of overlapping (blended) images. \\ 
saturated & bool & 1 &  & True if the peak pixel is saturated. \\ 
truncated & bool & 1 &  & True if the object is truncated (too close to an image boundary, defined as being within 4 arcsec from the CCD edge). \\ 
brightNeighb & bool & 1 &  & True if the object is within 10 arcmin of a star brighter than V < 4.5. \\ 
reliable & bool & 1 & & True if the source is stellar, not deblended, not saturated, not truncated, not near a bright neighbour, and has less than 1 bad pixel. \\

seeing & real & 4 & & Average seeing in the exposure. \\
mjd & real & 4 & & Modified Julian Date at start of exposure. \\

\end{longtable}
\end{center}

\newpage

\subsection{iphasSource}

\begin{center}
\begin{longtable}{llclp{9cm}}
\caption[short]{Definition of the \textbf{iphasSource} table.} \label{iphasSource} \\

\hline \textbf{Name} & \textbf{Type} & \textbf{Length} & \textbf{Unit} & \textbf{Description} \\ \hline
\endfirsthead

\hline \textbf{Name} & \textbf{Type} & \textbf{Length} & \textbf{Unit} & \textbf{Description} \\ \hline
\endhead

\hline \multicolumn{5}{r}{{\it Continued on next page}} \\ 
\endfoot

\hline \hline
\endlastfoot

sourceID & int & 8 &  & Unique identification number assigned by the merge algorithm. \\
name & char & 25 &  & Official designation of the form `IPHAS JHHMMSS.ss+DDMMSS.s'. \\
ra & real & 8 & degrees & J2000 right ascension with respect to the 2MASS PSC reference frame (which is consistent with ICRS to within 0.1 arcsec). \\
dec & real & 8 & degrees & J2000 declination with respect to the 2MASS PSC reference frame (which is consistent with ICRS to within 0.1 arcsec). \\
l & real & 8 & degrees & Galactic longitude converted from ra/dec (IAU 1958). \\
b & real & 8 & degrees & Galactic latitude converted from ra/dec (IAU 1958). \\
pos\_err & real & 4 & arcsec & Astrometric fit error (RMS). \\
mergedClassStat & real & 4 &  & Merged N(0,1) stellarness-of-profile statistic. \\
mergedClass & int & 2 &  & Combined class flag from three bands (1=galaxy, 0=noise, -1=star, -2=probableStar, -3=probableGalaxy, -9=saturated). \\
pStar & real & 4 &  & Probability that the source is a star. \\
pGalaxy & real & 4 &  & Probability that the source is a galaxy. \\
pNoise & real & 4 &  & Probability that the source is noise. \\
pSaturated & real & 4 &  & Probability that the source is saturated. \\

r & real & 4 & mag & Default r-band magnitude in the Vega system (using aperMag2, i.e. 2.3 arcsec aperture diameter). \\ 
rErr & real & 4 & mag & Uncertainty for r. Does not include systematic errors. \\ 
rPeakMag & real & 4 & mag & r-band magnitude in the Vega system (based on the peak pixel height.) \\ 
rPeakMagErr & real & 4 & mag & Uncertainty in rPeakMag. Does not include systematics. \\ 
rAperMag1 & real & 4 & mag & r-band magnitude in the Vega system (using aperMag1, i.e. 1.2 arcsec aperture diameter) \\ 
rAperMag1err & real & 4 & mag & Uncertainty in rAperMag1. Does not include systematics. \\ 
rAperMag3 & real & 4 & mag & r-band magnitude in the Vega system (using aperMag1, i.e. 3.3 arcsec aperture diameter) \\ 
rAperMag3err & real & 4 & mag & Uncertainty in rAperMag3. Does not include systematics. \\ 
rGauSig & real & 4 & pixels & RMS of axes of ellipse fit in r. \\ 
rEll & real & 4 & & Ellipticity in the r-band. \\
rPA & real & 4 & & Position angle in the r-band. \\
rDeblend & bool &  &  & True if the object had to be deblended in the r band, i.e. if it was part of overlapping images. \\ 
rClass & int & 2 & & Discrete image classification flag. \\
rClassStat & real & 4 & & N(0,1) stellarness-of-profile statistic. \\
rSaturated & bool &  &  & True if the peak pixel is saturated in the r-band. \\ 
rOnEdge & bool &  &  & True if the detection is located within 4 arcsec (12 pixels) from the edge of the CCD. \\ 
rReliable & bool & & & True if the source is stellar, not deblended, not saturated, not near the edge, and detected at SNR $>$ 5. \\
rSeeing & real & 4 & & Average seeing in the r-band exposure. \\
rDetectionID & int & 8 & & Reference to the r-band detection of this source in the iphasDetection table. \\


i & real & 4 & mag & Default i-band magnitude in the Vega system (using aperMag2, i.e. 2.3 arcsec aperture diameter). \\ 
rErr & real & 4 & mag & Uncertainty for i. Does not include systematic errors. \\ 
iPeakMag & real & 4 & mag & i-band magnitude in the Vega system (based on the peak pixel height.) \\ 
iPeakMagErr & real & 4 & mag & Uncertainty in iPeakMag. Does not include systematics. \\ 
iAperMag1 & real & 4 & mag & i-band magnitude in the Vega system (using aperMag1, i.e. 1.2 arcsec aperture diameter) \\ 
iAperMag1err & real & 4 & mag & Uncertainty in iAperMag1. Does not include systematics. \\ 
iAperMag3 & real & 4 & mag & i-band magnitude in the Vega system (using aperMag1, i.e. 3.3 arcsec aperture diameter) \\ 
iAperMag3err & real & 4 & mag & Uncertainty in iAperMag3. Does not include systematics. \\ 
iGauSig & real & 4 & pixels & RMS of axes of ellipse fit in i. \\ 
iEll & real & 4 & & Ellipticity in the i-band. \\
iPA & real & 4 & & Position angle in the i-band. \\
iDeblend & bool &  &  & True if the object had to be deblended in the i-band, i.e. if it was part of overlapping images. \\ 
iClass & int & 2 & & Discrete image classification flag. \\
iClassStat & real & 4 & & N(0,1) stellarness-of-profile statistic. \\
iSaturated & bool &  &  & True if the peak pixel is saturated in the i-band. \\ 
iOnEdge & bool &  &  & True if the detection is located within 4 arcsec (12 pixels) from the edge of the CCD. \\ 
iReliable & bool & & & True if the source is stellar, not deblended, not saturated, not near the edge, and detected at SNR $>$ 5. \\
iSeeing & real & 4 & & Average seeing in the i-band exposure. \\
iDetectionID & int & 8 & & Reference to the i-band detection of this source in the iphasDetection table. \\


ha & real & 4 & mag & Default H-alpha magnitude in the Vega system (using aperMag2, i.e. 2.3 arcsec aperture diameter). \\ 
rErr & real & 4 & mag & Uncertainty for ha. Does not include systematic errors. \\ 
iPeakMag & real & 4 & mag & H-alpha magnitude in the Vega system (based on the peak pixel height.) \\ 
iPeakMagErr & real & 4 & mag & Uncertainty in haPeakMag. Does not include systematics. \\ 
iAperMag1 & real & 4 & mag & H-alpha magnitude in the Vega system (using aperMag1, i.e. 1.2 arcsec aperture diameter) \\ 
iAperMag1err & real & 4 & mag & Uncertainty in haAperMag1. Does not include systematics. \\ 
iAperMag3 & real & 4 & mag & H-alpha magnitude in the Vega system (using aperMag1, i.e. 3.3 arcsec aperture diameter) \\ 
iAperMag3err & real & 4 & mag & Uncertainty in haAperMag3. Does not include systematics. \\ 
iGauSig & real & 4 & pixels & RMS of axes of ellipse fit in H-alpha. \\ 
iEll & real & 4 & & Ellipticity in H-alpha. \\
iPA & real & 4 & & Position angle in H-alpha. \\
iDeblend & bool &  &  & True if the object had to be deblended in H-alpha, i.e. if it was part of overlapping images. \\ 
iClass & int & 2 & & Discrete image classification flag. \\
iClassStat & real & 4 & & N(0,1) stellarness-of-profile statistic. \\
iSaturated & bool &  &  & True if the peak pixel is saturated in H-alpha. \\ 
iOnEdge & bool &  &  & True if the detection is located within 4 arcsec (12 pixels) from the edge of the CCD. \\ 
iReliable & bool & & & True if the source is stellar, not deblended, not saturated, not near the edge, and detected at SNR $>$ 5. \\
iSeeing & real & 4 & & Average seeing in the H-alpha exposure. \\
iDetectionID & int & 8 & & Reference to the H-alpha detection of this source in the iphasDetection table. \\

date & char & 10 & yyyy-mm-dd & Date of the night in which this source was 
observed. \\
fieldID & int & \tbd & & Unique identifier of the field observation. \\

r2 & real & 4 & mag & Alternative magnitude obtained in the same night as the default magnitude, if available. \\ 
i2 & real & 4 & mag & Alternative magnitude obtained in the same night as the default magnitude, if available. \\
ha2 & real & 4 & mag & Alternative magnitude obtained in the same night as the default magnitude, if available. \\ 
r2\_err & real & 4 & mag & Uncertainty for r2. Does not include systematic errors. \\ 
i2\_err & real & 4 & mag & Uncertainty for i2. Does not include systematic errors. \\
ha2\_err & real & 4 & mag & Uncertainty for ha2. Does not include systematic errors. \\ 
r2\_detectionID & \tbd & & & Identifier in the detection table. \\
i2\_detectionID & \tbd &  & & Identifier in the detection table. \\
ha2\_detectionID & \tbd & & & Identifier in the detection table. \\
nobs & int & 1 & & Total number of detections of this source available in the iphasDetection table. 
\\

\end{longtable}
\end{center}



\subsection{iphasRun}

Indexed by runID.

\tbd

\subsection{iphasField}

Indexed by fieldID.

\tbd

\section{TODO}
\begin{itemize}
\item How merge onEdge/Saturated/Reliable columns?
\item Show example queries
\item Bright star flag?
\item Columns for H-alpha are named i.
\item Change terminology: alternative magnitude into secondary magnitude.
\item Offset matching distances between bands, i.e. 4 columns i_dra, i_ddec, 
h_dra, h_ddec.  (UKIDSS has terminology... but maybe just iRA, haRA, etc is 
better?)
\item Promise a max time-interval for r2/i2/ha2, and promise that it's the 
partner.
\end{itemize}

\end{document}
